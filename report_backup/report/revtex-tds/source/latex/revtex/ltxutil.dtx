% \iffalse meta-comment balanced on line 107
% ltxutil.dtx: utilities package
% Copyright (c) 2009 Arthur Ogawa
%
% Disclaimer
%   This file is distributed WITHOUT ANY WARRANTY;
%   without even the implied warranty of
%   MERCHANTABILITY or FITNESS FOR A PARTICULAR PURPOSE.
% License
%   You may distribute this file under the conditions of the 
%   LaTeX Project Public License 1.3c or later 
%   (http://www.latex-project.org/lppl.txt).
% ReadMe
%   For the documentation and more detailed instructions for
%   installation, typeset this document with \LaTeX.
% Maintenance Status
%   This work has the LPPL maintenance status "maintained";
%   Current Maintainer of this work is Arthur Ogawa
%   changes for version 4.2d and 4.2e by Phelype Oleinik.
%
% This work consists of the main source file ltxutil.dtx
% and the derived files
%    ltxutil.sty, ltxutil.pdf
% Distribution:
%    CTAN:macros/latex/contrib/revtex/
%
% Unpacking:
%    tex ltxutil.dtx
%
% Documentation:
%    latex ltxutil.dtx; ...
%
%    Program calls to get the documentation (example):
%       pdflatex ltxutil.dtx
%       makeindex -s gind.ist ltxutil.idx
%       makeindex -s gglo.ist -o ltxutil.gls ltxutil.glo
%       pdflatex ltxutil.dtx
%       makeindex -s gind.ist ltxutil.idx
%       pdflatex ltxutil.dtx
%
% Installation:
%    TDS:doc/latex/revtex/
%    TDS:source/latex/revtex/
%    TDS:tex/latex/revtex/
%
% Thanks, Heiko!
%    This method of letting a single .dtx file serve as both
%    documentation (via latex) and installer (via tex) follows
%    the example of Heiko Oberdiek. Thanks!
%<*ignore>
\begingroup
  \def\x{LaTeX2e}%
\expandafter\endgroup
\ifcase
 0\expandafter\ifx\csname processbatchFile\endcsname\relax\else1\fi\ifx\fmtname\x\else 1\fi
 \relax
\else
 \csname fi\endcsname
%</ignore>
%<*install>
\input docstrip
\preamble

This is a generated file;
altering it directly is inadvisable;
instead, modify the original source file.
See the URL in the file README-LTXUTIL.tex.

License
   You may distribute this file under the conditions of the 
   LaTeX Project Public License 1.3c or later 
   (http://www.latex-project.org/lppl.txt).
   
   This file is distributed WITHOUT ANY WARRANTY; 
   without even the implied warranty of MERCHANTABILITY 
   or FITNESS FOR A PARTICULAR PURPOSE.

\endpreamble
\askforoverwritefalse
\keepsilent
 \generate{%
  %{ignore}
  \file{ltxutil.sty}{%
   \from{ltxutil.dtx}{package,kernel}%
  }%
  \file{ltxutil.krn}{%
   \from{ltxutil.dtx}{kernel}%
  }%
 }%
\ifToplevel{
\Msg{***********************************************************}
\Msg{*}
\Msg{* To finish the installation, please move}
\Msg{*    ltxutil.sty}
\Msg{* into a directory searched by TeX;}
\Msg{* in a TDS-compliant installation:}
\Msg{* texmf/tex/macros/latex/revtex/.}
\Msg{*}
\Msg{* To produce the documentation,
       run ltxutil.dtx through LaTeX.}
\Msg{*}
\Msg{* Happy TeXing}
\Msg{***********************************************************}
}
\endbatchfile
%</install>
%<*ignore>
\fi
%</ignore>
% \fi
%
% \GetFileInfo{ltxutil.dtx}
%
% \iffalse ltxdoc klootch
%<*package>
%%%  @LaTeX-file{
%%%     filename        = "ltxutil.dtx",
%%%     version         = "4.2e",
%%%     date            = "2020/10/03",
%%%     author          = "Arthur Ogawa (mailto:arthur_ogawa at sbcglobal.net),
%%%                        Phelype Oleinik (mailto:phelype.oleinik at latex-project.org),
%%%                        commissioned by the American Physical Society. Minor changes by Mark Doyle for version 4.2a-c.
%%%                        ",
%%%     copyright       = "Copyright (C) 1999, 2009 Arthur Ogawa,
%%%                        distributed under the terms of the 
%%%                        LaTeX Project Public License 1.3c, see
%%%                        ftp://ctan.tug.org/macros/latex/base/lppl.txt
%%%                        ",
%%%     address         = "Arthur Ogawa,
%%%                        USA",
%%%     telephone       = "",
%%%     FAX             = "",
%%%     email           = "mailto colon arthur_ogawa at sbcglobal.net",
%%%     codetable       = "ISO/ASCII",
%%%     keywords        = "latex, page grid, main vertical list",
%%%     supported       = "yes",
%%%     abstract        = "utilities package",
%%%  }
%</package>
% \fi
%
% \iffalse ltxdoc klootch
% The following references the \file{README-LTXUTIL} file,
% which contains basic information about this package.
% The contents of this file are generated when
% you typeset the programmer's documentation.
% Search on "{filecontents*}{README-LTXUTIL}" to locate it.
% \fi\input{README-LTXUTIL}%
%
% \subsection{Bill of Materials}
%
% Following is a list of the files in this distribution arranged
% according to provenance.
%
% \subsubsection{Primary Source}%
% One single file generates all.
%\begin{verbatim}
%ltxutil.dtx
%\end{verbatim}
%
% \subsubsection{Generated by \texttt{latex ltxutil.dtx}}%
% Typesetting the source file under pdflatex
% generates the readme and the documentation.
%\begin{verbatim}
%README-LTXUTIL ltxutil.pdf
%\end{verbatim}
%
% \subsubsection{Generated by \texttt{tex ltxutil.dtx}}%
% Typesetting this file with \TeX\ generates
% the package file.
%\begin{verbatim}
%ltxutil.sty
%\end{verbatim}
%
% \subsubsection{Auxiliary}%
% The following are auxiliary files generated
% in the course of running \LaTeX:
% \begin{verbatim}
%ltxutil.aux ltxutil.idx ltxutil.ind ltxutil.log ltxutil.toc
% \end{verbatim}
%
% \section{Code common to all modules}%
%
% We want to require only one place in this file
% where the version number is stated,
% and we also want to ensure that the version
% number is embedded into every generated file.
%
% Now we declare that
% these files can only be used with \LaTeXe.
% An appropriate message is displayed if
% a different \TeX{} format is used.
%    \begin{macrocode}
%<*doc|package>
\NeedsTeXFormat{LaTeX2e}[1995/12/01]%
%</doc|package>
%    \end{macrocode}
% As desired, the following modules all
% take common version information:
%    \begin{macrocode}
%<kernel&!package&!doc>\typeout{%
%<*package|doc>
\ProvidesFile{%
%</package|doc>
%<*kernel|package|doc>
ltxutil%
%</kernel|package|doc>
%<*doc>
.dtx%
%</doc>
%<package>.sty%
%<*package|doc>
}%
%</package|doc>
%    \end{macrocode}
%
% The following line contains, for once and for all,
% the version and date information.
% By various means, this information is reproduced
% consistently in all generated files and in the
% typeset documentation.
% Give credit where due. 
%    \begin{macrocode}
%<*doc|package|kernel>
%<version>
 [2020/10/03 4.2e utilities package (portions licensed from W. E. Baxter web at superscript.com)]% \fileversion
%</doc|package|kernel>
%<kernel&!package&!doc>}%
%    \end{macrocode}
%
%
% \section{The driver module \texttt{doc}}
%
% This module, consisting of the present section,
% typesets the programmer's documentation,
% generating the \file{README-LTXUTIL} as required.
%
% Because the only uncommented-out lines of code at the beginning of
% this file constitute the \file{doc} module itself,
% we can simply typeset the \file{.dtx} file directly,
% and there is thus rarely any need to
% generate the ``doc'' {\sc docstrip} module.
% Module delimiters are nonetheless required so that
% this code does not find its way into the other modules.
%
% The \enve{document} command concludes the typesetting run.
%
%    \begin{macrocode}
%<*doc>
%    \end{macrocode}
%
% \subsection{The Preamble}
% The programmers documentation is formatted
% with the \classname{ltxdoc} class with local customizations,
% and with the usual code line indexing.
%    \begin{macrocode}
\documentclass{ltxdoc}
\RequirePackage{ltxdocext}%
\let\url\undefined
\RequirePackage[colorlinks=true,linkcolor=blue]{hyperref}%
\pdfstringdefDisableCommands{%
  \let\file\relax
  \let\sc\relax
}
%\expandafter\ifx\csname package@font\endcsname\@undefined\else
% \expandafter\RequirePackage\expandafter{\csname package@font\endcsname}%
%\fi
\CodelineIndex\EnableCrossrefs % makeindex -s gind.ist ltxutil
\RecordChanges % makeindex -s gglo.ist -o ltxutil.gls ltxutil.glo
%    \end{macrocode}
%
% \subsubsection{Docstrip and info directives}
%    We use so many {\sc docstrip} modules that we set the
%    \texttt{StandardModuleDepth} counter to 1.
%    \begin{macrocode}
\setcounter{StandardModuleDepth}{1}
%    \end{macrocode}
%    The following command retrieves the date and version information
%    from this file.
%    \begin{macrocode}
\expandafter\GetFileInfo\expandafter{\jobname.dtx}%
%    \end{macrocode}
%
% \subsection{The ``Read Me'' File}
% As promised above, here is the contents of the
% ``Read Me'' file. That file serves a double purpose,
% since it also constitutes the beginining of the
% programmer's documentation. What better thing, after
% all, to have appear at the beginning of the
% typeset documentation?
%
% A good discussion of how to write a ReadMe file can be found in
% Engst, Tonya, ``Writing a ReadMe File? Read This''
% \emph{MacTech} October 1998, p. 58.
%
% Note the appearance of the
% \cmd\StopEventually\ command, which marks the
% dividing line between the user documentation
% and the programmer documentation.
%
% The usual user will not be asked to
% do a full build, not to speak
% of the bootstrap.
% Instructions for carrying out these procedures
% begin the programmer's manual.
%
%    \begin{macrocode}
\begin{filecontents*}{README-LTXUTIL}
\title{%
 A \LaTeX\ Package of utility macros%
 \thanks{%
  This file has version number \fileversion,
  last revised \filedate.%
 }%
 \thanks{%
  Version \fileversion\ \copyright\ 2019--2020 American Physical Society
 }%
}%
\author{%
 Arthur Ogawa%
 \thanks{\texttt{mailto:arthur\_ogawa at sbcglobal.net}}%
}%
%\iffalse
% For version number and date,
% search on "\fileversion" in the .dtx file,
% or see the end of the README-LTXUTIL file.
%\fi
\maketitle

This file embodies the \classname{ltxutil} package,
the implementation and its user documentation.

The distribution point for this work is
\url{journals.aps.org/revtex},
which contains prebuilt runtime files, documentation, and full source, 
ready to add to a TDS-compliant \TeX\ installation.

The \classname{ltxutil} package was commissioned by the American Physical Society
and is distributed under the terms of the \LaTeX\ Project Public License 1.3c,
the same license under which all the portions of \LaTeX\ itself are distributed.
Please see \url{http://ctan.tug.org/macros/latex/base/lppl.txt} for details.

To use this document class, you must have a working
\TeX\ installation equipped with \LaTeXe\ 
and possibly pdftex and Adobe Acrobat Reader or equivalent.

To install, retrieve the distribution,
unpack it into a directory on the target computer,
and move the file \file{ltxutil.sty}
into a location in your filesystem where it will be found by \LaTeX.

To use, read the user documentation \file{ltxutil.pdf}.

\tableofcontents

\section{Processing Instructions}

The package file \file{ltxutil.sty}
is generated from this file, \file{ltxutil.dtx},
using the {\sc docstrip} facility of \LaTeX
via |tex ltxutil.dtx| (Note: do \emph{not} use \LaTeX\ for this task).
The typeset documentation that you are now reading is generated from
the same file by typesetting it with \LaTeX\ or pdftex
via |latex ltxutil.dtx| or |pdflatex ltxutil.dtx|.

\subsection{Build Instructions}

You may bootstrap this suite of files solely from \file{ltxutil.dtx}.
Prepare by installing \LaTeXe\ (and either tex or pdftex) on your computer,
then carry out the following steps:
\begin{enumerate}
\item
Within an otherwise empty directory,
typeset \file{ltxutil.dtx} with \LaTeX\ or pdflatex;
you will obtain the typeset documentation you are now reading,
along with the file \file{README-LTXUTIL}.

Note: you will have to run \LaTeX, then 
\file{makeindex} \texttt{-s gind.ist ltxutil.idx}, then
\file{makeindex} \texttt{-s gglo.ist -o ltxutil.gls ltxutil.glo}, then
\LaTeX\ again in order to obtain a valid index and table of contents.
\item
Now typeset \file{ltxutil.dtx} with \TeX (not \LaTeX),
thereby generating the package file \file{ltxutil.sty}.
\item
Install the following files into indicated locations within your 
TDS-compliant \texttt{texmf} tree (you may need root access):
\begin{itemize}
\item
\file{$TEXMF/}\file{tex/}\file{latex/}\file{revtex/}\classname{ltxutil.sty}
\item
\file{$TEXMF/}\file{source/}\file{latex/}\file{revtex/}\classname{ltxutil.dtx}
\item
\file{$TEXMF/}\file{doc/}\file{latex/}\file{revtex/}\classname{ltxutil.pdf}
\end{itemize}
where \file{$TEXMF/} stands for \file{texmf-local/}, or some other \texttt{texmf} tree
in your installation.
\item
Run \texttt{mktexlsr} on \file{$TEXMF/} (you may need root access).
\item
Build and installation are now complete; 
now put a \cmd\usepackage\texttt{\{ltxutil\}} in your document preamble!
\end{enumerate}

\subsection{Change Log}
\changes{4.0b}{1999/06/20}{AO: Fixed spurious \texttt{CR} and (return) characters in output file. Also, if the document did not have the \cs{end}\texttt{figure} on a line of its own, the macro wouldn't work. Fixed.}
\changes{4.0b}{1999/06/20}{AO: Removed superfluous \cs{def}s, changed to using \cs{floats@sw} as the flag. Also stopped using DPC's \cs{if@twocolumn} flag: using \cs{floats@sw} instead. Also added \cs{par}\cs{vskip}\cs{z@skip} after the \cs{minipagefootnotes} so that the float box would have zero depth like the kernel one. }
\changes{4.0b}{1999/06/20}{only execute if there really were floats of the given type}
\changes{4.0b}{1999/06/20}{Support the hack with \cs{prepdef}, and delay until \cs{AtBeginDocument} time, since \classname{hyperref} clobbers \cs{caption}.}
\changes{4.0c}{1999/11/13}{(AO, 110) Install hooks for endfloats processing}
\changes{4.0c}{1999/11/13}{(AO, 116) Hyperref compatibility}
\changes{4.0c}{1999/11/13}{(AO, 130) Interference from array package}
\changes{4.0c}{1999/11/13}{*-form mandates pagebreak at each float; only print section head if there is something there.}
\changes{4.0d}{2000/04/10}{(AO, 127) Floats placed [h] to allow page breaks}
\changes{4.0d}{2000/04/10}{(AO, 174) kernel fix}
\changes{4.0d}{2000/05/19}{(AO, 224) Hyperref compatibility.}
\changes{4.0d}{2000/05/23}{Allow things to break over pages by setting array@default.}
\changes{4.0e}{2000/11/16}{(AO, 221) Remove samepage command from @xfloat@prep: If the float can break over pages, we want better control.}
\changes{4.0f}{2001/07/13}{(AO, 404) Hyperref compatibility}
\changes{4.1a}{2008/01/19}{(AO, 459) do not assume \cs{class@name} is defined}%
\changes{4.1a}{2008/01/19}{(AO, 461) Change the csname from \cs{@dotsep} to \cs{ltxu@dotsep}. The former is understood in mu. (What we wanted was a dimension.)}%
\changes{4.1a}{2008/01/19}{(AO, 475) I had not properly reproduced the LaTeX macro \cs{eqnarray}.}%
\changes{4.1a}{2008/01/19}{(AO, 479) Per: Dylan Thurston<dpt at math.harvard.edu>}%
\changes{4.1a}{2008/06/30}{(AO) Make \cs{addtocontents} a \cs{long} \cs{def}; gobble up \cs{footnote}}%
\changes{4.1a}{2008/06/30}{(AO) Remove code that avoided changes to \cs{@xfootnotemark}}%
\changes{4.1a}{2008/06/30}{(AO, 438) Complete rewrite of footnote macros.}
\changes{4.1a}{2008/07/07}{\cs{@xfloat@prep} calls \cs{ltx@footnote@pop} to restore the original \cs{ltx@footmark} and \cs{ltx@foottext} procedures, in case footnote processing has switched.}
\changes{4.1a}{2008/08/12}{\cs{class@documenthook} is the last \cs{AtBeginDocument} token now}
\changes{4.1a}{2008/08/12}{Class extension mechanism \cs{@pushfilename@ltx} and \cs{@p@pfilename@ltx}.}
\changes{4.1a}{2008/08/12}{Class extension mechanism \cs{class@extension}, \cs{class@extensionfile}, and \cs{class@ext@hook}.}
\changes{4.1a}{2008/08/12}{Get rid of \cs{set@typesize@hook} \cs{set@pica@hook} and the \cs{normalsize} directive}%
\changes{4.1b}{2008/08/12}{(AO, 487) Support for video figures and the \cs{setfloatlink} command}%
\changes{4.1b}{2008/08/12}{(AO, 505) try to accommodate \classname{colortbl}.}
\changes{4.1b}{2008/08/12}{Acquire \classname{hyperref} savoire}
\changes{4.1b}{2008/08/12}{Default assignment of \cs{float@sw} now, not at \cs{AtBeginDocument} time.}%
\changes{4.1b}{2008/08/12}{If class option \classoption{lengthcheck} is in effect, log the height of this float class.}
\changes{4.1b}{2008/08/12}{No need to protect against undefined \cs{float@sw}}
\changes{4.1b}{2008/08/12}{Patch the array package even later: after all package patches go in.}
\changes{4.1b}{2008/08/12}{Refine toc processing: provide default.}%
\changes{4.1b}{2008/08/12}{Tally and log the height of a float class}
\changes{4.1d}{2009/03/27}{(AO, 511) Compatability with lineno.sty's erroneous way of detecting fleqn.clo}%
\changes{4.1f}{2009/07/07}{(AO, 515) Hook for setting the font of a footnote}
\changes{4.1f}{2009/07/10}{(AO, 518) Tally register overflow when locument is long}
\changes{4.1g}{2009/10/06}{(AO, 532) Both arguments of \cs{href} get sanitized}%
\changes{4.1g}{2009/10/07}{(AO, 525) Remove phantom paragraph above display math that is given in vertical mode}%
\changes{4.1g}{2009/10/07}{(AO, 539) Use of double-backslash in argument of \cs{section} gives error. The \classname{textcase} package is involved.}%
\changes{4.1n}{2009/12/05}{(AO, 569) Use of \classname{hyperref} interferes with column balancing of last page}%
\changes{4.1n}{2009/12/06}{(AO) Incorporate change to ltmiscen.dtx v1.1i 2000/05/19}%
\changes{4.1n}{2009/12/09}{(AO, 569) execute \classname{atveryend}'s \cs{Call@AfterLastShipout} at the proper time}%
\changes{4.1n}{2009/12/13}{(AO, 574) protect against \classname{lineno.sty}, which forces a visit to the output routine, which appears to destroy the value of \cs{@tempdima}}%
\changes{4.1n}{2010/01/02}{(AO, 571) Interface \cs{set@footnotewidth} for determining the set width of footnotes}%
\changes{4.1n}{2010/01/02}{(AO, 571) allow split after last line of footnote}%
\changes{4.1n}{2010/01/06}{(AO, 572) title block footnotes numbered independently from body footnotes}%
\changes{4.1p}{2010/02/24}{(AO, 582) A patch of \classname{hyperref.sty} to provide backward compatibility to \TeX Live 2007's version 6.75r}%
\changes{4.2a}{2017/11/21}{(MD) Use updated best practice to use https and doi.org}%
\changes{4.2a}{2018/12/12}{(MD) Updated name of README file and use standard fonts when typesetting}%
\changes{4.2d}{2020/09/19}{(PHO) Adapt \cs{document} and \cs{enddocument} hooks to the 2020-10-01 \LaTeX{} release.}%

\end{filecontents*}
%    \end{macrocode}
%
% \subsection{The Document Body}
%
% Here is the document body, containing only a
% \cmd\DocInput\ directive---referring to this very file.
% This very cute self-reference is a common \classname{ltxdoc} idiom.
%    \begin{macrocode}
\begin{document}%
\expandafter\DocInput\expandafter{\jobname.dtx}%
\end{document}
%    \end{macrocode}
%
%    \begin{macrocode}
%</doc>
%    \end{macrocode}
%
% \section{Using this package}
% Once this package is installed on your filesystem, you can employ it in
% adding functionality to \LaTeX\ by invoking it in your document or document class.
%
% \subsection{Invoking the package}
% In your document, you can simply call it up in your preamble:
% \begin{verbatim}
%\documentclass{book}%
%\usepackage{ltxutil}%
%\begin{document}
%<your document here>
%\end{document}\end{verbatim}
% However, the preferred way is to invoke this package from within your
% customized document class:
% \begin{verbatim}
%\NeedsTeXFormat{LaTeX2e}[1995/12/01]%
%\ProvidesClass{myclass}%
%\RequirePackage{ltxutil}%
%\LoadClass{book}%
%<class customization commands>
%\endinput\end{verbatim}
%
% Once loaded, the package gives you acccess to certain procedures,
% usually to be invoked by a \LaTeX\ command or environment, but not at the document level.
%
%
% \section{Compatibility with \LaTeX's Required Packages}
% Certain packages, usually ones written by members of the 
% \LaTeX\ Project itself, have been designated ``required'' and
% are distributed as part of standard \LaTeX.
% These packages have been placed in a priviledged position
% vis \'a vis the \LaTeX\ kernel in that they override the definitions of certain kernel macros.
%
% The \classname{ltxutil} package will be incompatible with any package that
% redefines any of the kernel macros that \classname{ltxutil} patches---if that
% package is loaded \emph{after} \classname{ltxutil}. This means that for
% greatest compatibility, \classname{ltxutil} should be loaded \emph{after},
% say, \classname{ftnright}, which overwrites \LaTeX's kernel
% procedures \cmd\@outputdblcol, \cmd\@startcolumn, and \cmd\@makecol.
%
% Hereinafter follows some notes on specific \LaTeX\ packages.
%
% \subsection{array}
% This package alters the way tabular environments are done, 
% therefore it could run afoul of the \LaTeX\ ``required'' package \classname{array} or any
% package that calls for it to be loaded.
% However, this package has provisions for remaining compatible with \classname{array}.
% So long as the version of \classname{array} that is used with this package has the appropriate
% meanings for the procedures it overwrites, all should be well.
%
% \subsection{longtable}
% David Carlisle's \classname{longtable} package modifies both the \LaTeX\ kernel and the 
% \classname{array} package. This package must therefore alter \cmd\LT@array. 
% For now, that job is handled by \classname{ltxgrid}.
% 
%
%\StopEventually{}
%
% \section{Implementation of package}
%
% Special acknowledgment: this package uses concepts pioneered
% and first realized by William Baxter (mailto:web at superscript.com)
% in his SuperScript line of commercial typesetting tools, and
% which are used here with his permission. 
%
% \subsection{Beginning of the \file{package} {\sc docstrip} module}
%    \begin{macrocode}
%<*package>
\def\package@name{ltxutil}%
\expandafter\PackageInfo\expandafter{\package@name}{%
 Utility macros for \protect\LaTeXe,
 by A. Ogawa (arthur_ogawa at sbcglobal.net)%
}%
%</package>
%    \end{macrocode}
%
% \subsection{Banner and beginning of the \file{kernel} {\sc docstrip} module}%
%    \begin{macrocode}
%<*kernel>
%    \end{macrocode}
%
% \subsection{Errors and warnings}
%
% \begin{macro}{\class@err}
% \begin{macro}{\class@warn}
% \begin{macro}{\class@info}
% A few shorthands for Class messages.
% Your document class should define \cmd\class@name.
%    \begin{macrocode}
\def\class@err#1{\ClassError{\class@name}{#1}\@eha}%
\def\class@warn#1{\ClassWarningNoLine{\class@name}{#1}}%
\def\class@info#1{\ClassInfo{\class@name}{#1}}%
\def\obsolete@command#1{%
 \class@warn@end{Command \string#1\space is obsolete.^^JPlease remove from your document}%
 \global\let#1\@empty
 #1%
}%
\def\replace@command#1#2{%
 \class@warn@end{Command \string#1\space is obsolete;^^JUse \string#2\space instead}%
 \global\let#1#2%
 #1%
}%
\def\replace@environment#1#2{%
 \class@warn@end{Environment #1 is obsolete;^^JUse #2 instead}%
 \glet@environment{#1}{#2}%
 \@nameuse{#1}%
}%
\def\incompatible@package#1{%
 \@ifpackageloaded{#1}{%
  \def\@tempa{I cannot continue. You must remove the \string\usepackage\ statement that caused that package to be loaded.}%
  \ClassError{\class@name}{The #1 package cannot be used with \class@name}%
  \@tempa\stop
 }{%
  \class@info{#1 was not loaded (OK!)}%
 }%
}%
\def\class@warn@end#1{%
 \gappdef\class@enddocumenthook{\class@warn{#1}}%
}%
%    \end{macrocode}
%    \end{macro}
%    \end{macro}
%    \end{macro}
%
% \changes{4.1a}{2008/01/19}{(AO, 459) do not assume \cs{class@name} is defined}%
% Give \cmd\class@name\ a meaning if it does not already have one. 
%    \begin{macrocode}
\ifx\undefined\class@name
 \def\class@name{ltxutil}%
 \class@warn{You should define the class name before reading in this package. Using default}%
\fi
%    \end{macrocode}
%
% \subsection{New Tools}%
%
% \begin{macro}{\t@}
%    \begin{macrocode}
\def\t@{to}%
%    \end{macrocode}
% \end{macro}
%
% \begin{macro}{\dimen@iii}
%    \begin{macrocode}
\dimendef\dimen@iii\thr@@
%    \end{macrocode}
% \end{macro}
%
% \begin{macro}{\halignt@}
%    \begin{macrocode}
\def\halignt@{\halign\t@}%
%    \end{macrocode}
% \end{macro}
%
% \begin{macro}{\f@ur}
% Analogous to \cmd\@ne, \cmd\tw@, and \cmd\thr@@.
%    \begin{macrocode}
\chardef\f@ur=4\relax
\chardef\cat@letter=11\relax
\chardef\other=12\relax
%    \end{macrocode}
% \end{macro}
%
% \begin{macro}{\let@environment}
% \begin{macro}{\glet@environment}
% The directive \cmd\let@environment\ takes care of a common programming
% idiom whereby one environment is made a synonym for another.
%    \begin{macrocode}
\def\let@environment#1#2{%
 \expandafter\let
 \csname#1\expandafter\endcsname\csname#2\endcsname
 \expandafter\let
 \csname end#1\expandafter\endcsname\csname end#2\endcsname
}%
\def\glet@environment#1#2{%
 \global\expandafter\let
 \csname#1\expandafter\endcsname\csname#2\endcsname
 \global\expandafter\let
 \csname end#1\expandafter\endcsname\csname end#2\endcsname
}%
%    \end{macrocode}
% \end{macro}
% \end{macro}
%
% \begin{macro}{\tracingplain}
% The command \cmd\tracingplain\ causes \TeX's tracing parameters to
% return to the values set by default. This command is sometimes
% useful when you have said \cmd\tracingall\ somewhere and want to
% restore.
% The \cmd\traceoutput\ command causes \cmd\tracingoutput\ diagnostics
% upon \cmd\shipout.
%    \begin{macrocode}
\newcommand\tracingplain{%
 \tracingonline\z@\tracingcommands\z@\tracingstats\z@
 \tracingpages\z@\tracingoutput\z@\tracinglostchars\@ne
 \tracingmacros\z@\tracingparagraphs\z@\tracingrestores\z@
 \showboxbreadth5\showboxdepth3\relax %\errorstopmode
 }%
\newcommand\traceoutput{%
 \appdef\@resetactivechars{\showoutput}%
}%
%    \end{macrocode}
% \end{macro}
%
% \begin{macro}{\say}
% \begin{macro}{\saythe}
% The commands \cmd\say\ and \cmd\saythe\ cause diagnostic messages in the
% \TeX\ log that give the value of a control sequence name or a register
% respectively.
%    \begin{macrocode}
\newcommand\say[1]{\typeout{<\noexpand#1=\meaning#1>}}%
\newcommand\saythe[1]{\typeout{<\noexpand#1=\the#1>}}%
%    \end{macrocode}
% \end{macro}
% \end{macro}
%
% \begin{macro}{\fullinterlineskip}
% Resets the \cmd\prevdepth\ so that the full amount of \cmd\baselineskip\ glue will be inserted by
% the \cmd\baselinesklip\ mechanism. 
% Can be invoked just after a \cmd\hrule\ to undo its default suppression of base line skip.
%    \begin{macrocode}
\def\fullinterlineskip{\prevdepth\z@}%
%    \end{macrocode}
% \end{macro}
%
% \begin{macro}{\count@i}
% \begin{macro}{\count@ii}
% 
%    \begin{macrocode}
\countdef\count@i\@ne
\countdef\count@ii\tw@
%    \end{macrocode}
% \end{macro}
% \end{macro}
%
%
% \subsection{Boolean Control}%
% We introduce just enough of the Boolean calculus for \TeX.
% Alan Jeffrey was the pioneer here, with an article in TUGboat
% (Vol. 11, No. 2, page 237).
% This implementation owes a debt to
% William Baxter (web at superscript.com).
% See articles by Baxter and Ogawa in the proceedings of the
% 1994 TUG meeting, TUGboat Vol.~15, No.~3.
%
% \begin{macro}{\prepdef}
% \begin{macro}{\appdef}
% \begin{macro}{\gappdef}
%
% Provide the capability of performing head- and tail patches.
% The procedure \cmd\prepdef\ prepends to the given macro
% the tokens specified in its second argument.
% Likewise for \cmd\appdef, except that it appends.
% Note that the first 10 toks registers are utility registers,
% and we simply make a control sequence name, \cmd\toks@ii, for one of
% them.
%    \begin{macrocode}
\long\def\prepdef#1#2{%
 \@ifxundefined#1{\toks@{}}{\toks@\expandafter{#1}}%
 \toks@ii{#2}%
 \edef#1{\the\toks@ii\the\toks@}%
}%
\long\def\appdef#1#2{%
 \@ifxundefined#1{\toks@{}}{\toks@\expandafter{#1}}%
 \toks@ii{#2}%
 \edef#1{\the\toks@\the\toks@ii}%
}%
\long\def\gappdef#1#2{%
 \@ifxundefined#1{\toks@{}}{\toks@\expandafter{#1}}%
 \toks@ii{#2}%
 \global\edef#1{\the\toks@\the\toks@ii}%
}%
\long\def\appdef@val#1#2{%
 \appdef#1{{#2}}%
}%
\long\def\appdef@e#1#2{%
 \expandafter\appdef
 \expandafter#1%
 \expandafter{#2}%
}%
\long\def\appdef@eval#1#2{%
 \expandafter\appdef@val
 \expandafter#1%
 \expandafter{#2}%
}%
\toksdef\toks@ii=\tw@
%    \end{macrocode}
% \end{macro}
% \end{macro}
% \end{macro}
%
% \begin{macro}{\@ifxundefined}
% \begin{macro}{\@ifnotrelax}
% \begin{macro}{\@argswap}
% \begin{macro}{\@argswap@val}
%
% Certain utility procedures use \cmd\@ifxundefined,
% which is defined here in terms of \cmd\@ifx.
% Others use \cmd\@ifnotrelax, namely when 
% the control sequence name is manufactured by
% the use of \cmd\csname.
%
% The procedures \cmd\@argswap and \cmd\@argswap@val
% are used to facilitate control of expansion.
%
%    \begin{macrocode}
\long\def\@ifxundefined#1{\@ifx{\undefined#1}}%
\long\def\@ifnotrelax#1#2#3{\@ifx{\relax#1}{#3}{#2}}%
\long\def\@argswap#1#2{#2#1}%
\long\def\@argswap@val#1#2{#2{#1}}%
\def\@ifxundefined@cs#1{\expandafter\@ifx\expandafter{\csname#1\endcsname\relax}}%
%    \end{macrocode}
% \end{macro}
% \end{macro}
% \end{macro}
% \end{macro}
%
% \begin{macro}{\rvtx@ifformat@geq}
%   Some changes in the \LaTeX{} kernel requires us to conditionally
%   define some macros depending on the version of the kernel.
%   \cmd\rvtx@ifformat@geq{} will check if the release date of the
%   currently-running \LaTeXe{} kernel is greater or equal to the
%   argument (the argument should be in the format \texttt{yyyy-mm-dd}).
% \changes{4.2d}{2020/09/17}{(PHO) Add \cs{rvtx@ifformat@geq}.}%
%    \begin{macrocode}
\ifx\IfFormatAtLeastTF\undefined
  \def\rvtx@ifformat@geq{\@ifl@t@r\fmtversion}%
\else
  \let\rvtx@ifformat@geq\IfFormatAtLeastTF
\fi
%    \end{macrocode}
% \end{macro}
%
% \begin{macro}{\@boolean}
% \begin{macro}{\@boole@def}
% In order to define \cmd\@ifx, we first must create the
% ``defining word'' (term taken form our Forth vocabulary)
% \cmd\@boole@def, which employs \cmd\@boolean\ to do its job.
%    \begin{macrocode}
\def\@boolean#1#2{%
  \long\def#1{%
    #2% \if<something>
      \expandafter\true@sw
    \else
      \expandafter\false@sw
    \fi
  }%
}%
\def\@boole@def#1#{\@boolean{#1}}% Implicit #2
%    \end{macrocode}
% \end{macro}
% \end{macro}
%
% \begin{macro}{\@booleantrue}
% \begin{macro}{\@booleanfalse}
% The procedures \cmd\@booleantrue\ and
% \cmd\@booleanfalse\ are assignment operators
% for Boolean flags.
%    \begin{macrocode}
\def\@booleantrue#1{\let#1\true@sw}%
\def\@booleanfalse#1{\let#1\false@sw}%
%    \end{macrocode}
% \end{macro}
% \end{macro}
%
% \begin{macro}{\@ifx}
% \begin{macro}{\@ifx@empty}
% \begin{macro}{\@if@empty}
% \begin{macro}{\@ifcat}%
% \begin{macro}{\@ifdim}%
% \begin{macro}{\@ifeof}%
% \begin{macro}{\@ifhbox}%
% \begin{macro}{\@ifhmode}%
% \begin{macro}{\@ifinner}%
% \begin{macro}{\@ifmmode}%
% \begin{macro}{\@ifnum}%
% \begin{macro}{\@ifodd}%
% \begin{macro}{\@ifvbox}%
% \begin{macro}{\@ifvmode}%
% \begin{macro}{\@ifvoid}%
% We can now invoke the defining word to create
% the procedures \cmd\@ifx\ and friends.
%
% Compatibility Note: earlier versions of this package 
% defined a procedure \cmd\@ifempty. However, for compatibility with AMS\LaTeX,
% we must avoid the following three names:
% \cmd\@ifempty, \cmd\@xifempty, and \cmd\@ifnotempty.
%
%    \begin{macrocode}
\@boole@def\@ifx#1{\ifx#1}%
\@boole@def\@ifx@empty#1{\ifx\@empty#1}%
\@boole@def\@if@empty#1{\if!#1!}%
%\@boole@def\@if@sw#1{\csname if#1\endcsname}%
\def\@if@sw#1#2{#1\expandafter\true@sw\else\expandafter\false@sw#2}%
\@boole@def\@ifdim#1{\ifdim#1}%
\@boole@def\@ifeof#1{\ifeof#1}%
\@boole@def\@ifhbox#1{\ifhbox#1}%
\@boole@def\@ifhmode{\ifhmode}%
\@boole@def\@ifinner{\ifinner}%
\@boole@def\@ifmmode{\ifmmode}%
\@boole@def\@ifnum#1{\ifnum#1}%
\@boole@def\@ifodd#1{\ifodd#1}%
\@boole@def\@ifvbox#1{\ifvbox#1}%
\@boole@def\@ifvmode{\ifvmode}%
\@boole@def\@ifvoid#1{\ifvoid#1}%
%    \end{macrocode}
% \end{macro}
% \end{macro}
% \end{macro}
% \end{macro}
% \end{macro}
% \end{macro}
% \end{macro}
% \end{macro}
% \end{macro}
% \end{macro}
% \end{macro}
% \end{macro}
% \end{macro}
% \end{macro}
% \end{macro}
%
% \begin{macro}{\true@sw}
% \begin{macro}{\false@sw}
%
% Note that when a Boolean operator expands, it
% employs two macros that act as selectors, defined here.
%
%    \begin{macrocode}
\long\def\true@sw#1#2{#1}%
\long\def\false@sw#1#2{#2}%
%    \end{macrocode}
% \end{macro}
% \end{macro}
%
% \begin{macro}{\loopuntil}
% \begin{macro}{\loopwhile}
%
% Loop control using the Boolean idiom.
% Superior to \cmd\loop\dots\cmd\repeat\ because these can be nested.
% The tail of the argument must have a Boolean predicate.
%
%    \begin{macrocode}
\long\def\loopuntil#1{#1{}{\loopuntil{#1}}}%
\long\def\loopwhile#1{#1{\loopwhile{#1}}{}}%
%    \end{macrocode}
% \end{macro}
% \end{macro}
%
% \begin{macro}{\@provide}
%
% A defining word that refuses to clobber a prior meaning.
%
%    \begin{macrocode}
\def\@provide#1{%
 \@ifx{\undefined#1}{\true@sw}{\@ifx{\relax#1}{\true@sw}{\false@sw}}%
 {\def#1}{\def\j@nk}%
}%
%    \end{macrocode}
% \end{macro}
%
%
% \subsection{Begin Document Structure}
% The standard \LaTeX\ mechanism \cmd\AtBeginDocument\
% is inadequate because the \cmd\vsize\ is bound much too early.
% We supply here a mechanism whereby decisions about the 
% page layout can be deferred until \cmd\AtBeginDocument\ time.
%
% The problem we are working around is that the \cmd\AtBeginDocument\
% hook in \cmd\document\ appears long after the calculation of
% \cmd\vsize\ and \cmd\hsize, that is, \LaTeX\ provides no mechanism
% for deferring the decision about the page grid until \cmd\AtBeginDocument\ time.
% We fix things by prepending a hook at the very beginning of \cmd\document.
%
% As it turns out, though, it appears feasible to simply invoke the desired 
% column grid command at \cmd\AtBeginDocument\ time, since the MVL has nothing in it
% at that time that would be problematical.
%
% \begin{macro}{\document}
% We begin by installing hooks into \cmd\document\ that
% we will manage ourselves.
%
% The 2020-10-01 \LaTeX{} release got a new hook management system and
% several new hooks (several previously provided by \textsf{etoolbox}).
% The one we want here is \texttt{begindocument/before}, the first thing
% executed by \cmd\document{}, right after ending the group started by
% \cmd\begin{}.
%
% Thus, if the \LaTeX{} kernel date is 2020-10-01 we just add to that
% hook, otherwise resort to the old method, patching \cmd\document:
% end the group started by \cmd\begin, apply our hook, and
% conclude our shenanigans by absorbing
% the first token of the expansion of \cmd\document, which
% we assume to be \cmd\endgroup{} (true until the aforementioned release).
% \changes{4.1a}{2008/08/12}{Get rid of \cs{set@typesize@hook} \cs{set@pica@hook} and the \cs{normalsize} directive}%
% \changes{4.2d}{2020/09/17}{(PHO) Use \LaTeX's hook management system, if possible.}%
%    \begin{macrocode}
\rvtx@ifformat@geq{2020-10-01}%
  {%
    \AddToHook{begindocument/before}{\document@inithook}%
  }{%
    \prepdef\document{%
     \endgroup
     \document@inithook
     \true@sw{}%
    }%
  }
%    \end{macrocode}
% \end{macro}
%
% \begin{macro}{\document@inithook}
% To use, simply \cmd\appdef\cmd\document@inithook\arg{your tokens here}.
%    \begin{macrocode}
\let\document@inithook\@empty
%    \end{macrocode}
% \end{macro}
%
% \begin{macro}{\class@documenthook}
% \begin{macro}{\class@enddocumenthook}
% \changes{4.1a}{2008/08/12}{\cs{class@documenthook} is the last \cs{AtBeginDocument} token now}
% We install the last \cmd\AtBeginDocument\ hook, namely the 
% procedure \cmd\class@documenthook. Within the document class,
% we will use this hook exclusively, so as to avoid interference from other packages.
% Similarly with \cmd\class@enddocumenthook, installed via \cmd\AtEndDocument.
%
% A document class using this package should do as this package does and just say, 
% \cmd\appdef\ \cmd\class@documenthook\ instead of \cmd\AtBeginDocument,
% and \cmd\appdef\ \cmd\class@enddocumenthook\ instead of \cmd\AtEndDocument.
%    \begin{macrocode}
\appdef\document@inithook{%
 \AtBeginDocument{\class@documenthook}%
}%
\AtEndDocument{%
 \class@enddocumenthook
}%
\let\class@documenthook\@empty
\let\class@enddocumenthook\@empty
%    \end{macrocode}
% \end{macro}
% \end{macro}
%
% \begin{macro}{\enddocument}
% \begin{macro}{\check@aux}
% \begin{macro}{\do@check@aux}
% \changes{4.1n}{2009/12/05}{(AO, 569) Use of \classname{hyperref} interferes with column balancing of last page}%
% The standard \LaTeX\ \enve{document} processing is a potential problem,
% particularly when the output routine has been changed by \classname{ltxgrid}.
% We separate out the procedure that checks the auxiliary file at the end of
% the job so that later it can be called from the safety of the output
% routine.
% We will do this to ensure that the \cmd\@mainaux\ stream is not closed until
% the last page of the job is shipped out, and that can only be done by coordinating
% with the output routine.
%
% \changes{4.2d}{2020/09/17}{(PHO) Only redefine \cs{enddocument} in older versions.}%
% This approach, however, will only be done for older versions of the
% \LaTeX{} kernel:
%    \begin{macrocode}
\rvtx@ifformat@geq{2020-10-01}{%
  % <definitions for newer LaTeX later>
}{%
  % <definitions for older LaTeX>
\def\enddocument{%
%    \end{macrocode}
% \changes{4.1n}{2009/12/06}{(AO) Incorporate change to ltmiscen.dtx v1.1i 2000/05/19}%
% The following line from \filename{ltmiscen.dtx} `resets \cmd\AtEndDocument for latex/3060'.
%    \begin{macrocode}
 \let\AtEndDocument\@firstofone
%    \end{macrocode}
%    \begin{macrocode}
 \@enddocumenthook
 \@checkend{document}%
%    \end{macrocode}
% The \cmd\clear@document\ statement ends the current page (we must guarantee no further shipouts),
% then executes all cleanup procedures that must occur only after the last shipout.
% Clients will queue up their procedures via \cmd\AfterLastShipout, if it exists, otherwise by doing \cmd\appdef\cmd\clear@document. 
%    \begin{macrocode}
 \clear@document
%    \end{macrocode}
% We are very close to ending the \TeX\ run, now.
%    \begin{macrocode}
 \check@aux
 \deadcycles\z@
 \@@end
}%
\def\check@aux{\do@check@aux}%
\def\do@check@aux{%
 \@if@sw\if@filesw\fi{%
  \immediate\closeout\@mainaux
  \let\@setckpt\@gobbletwo
  \let\@newl@bel\@testdef
  \@tempswafalse
  \makeatletter
  \input\jobname.aux\relax
 }{}%
 \@dofilelist
 \@ifdim{\font@submax >\fontsubfuzz\relax}{%
  \@font@warning{%
   Size substitutions with differences\MessageBreak
   up to \font@submax\space have occured.\@gobbletwo
  }%
 }{}%
 \@defaultsubs
 \@refundefined
 \@if@sw\if@filesw\fi{%
  \@ifx{\@multiplelabels\relax}{%
   \@if@sw\if@tempswa\fi{%
    \@latex@warning@no@line{%
     Label(s) may have changed.
     Rerun to get cross-references right%
    }%
   }{}%
  }{%
    \@multiplelabels
  }%
 }{}%
}%
}
%    \end{macrocode}
%
% \changes{4.2d}{2020/09/17}{(PHO) Patch \cs{enddocument} at runtime in newer versions.}%
% \begin{macro}{\rvtx@enddocument@patch}
%   For newer \LaTeX{} we'll try to be a bit more future-proof
%   (no miracle though).  The code for \cmd\enddocument{}
%   (in pre-2020-10-01 \LaTeX) is roughly:
%   \begin{verbatim}
%   \def\enddocument{%
%     <hooks and bookkeeping>
%     \clearpage
%     <read main .aux and final checks>
%     \@@end
%   }
%   \end{verbatim}
%   and the patches above replace the \cmd\clearpage{} by its own
%   \cmd\clear@document, and \verb|<read main .aux and final checks>| by
%   \cmd\do@check@aux, which it can later control the timing.
%
%   Now we will apply the same changes, but this time without redefining
%   \cmd\enddocument:  we will instead replace tokens on-the-fly, when
%   \cmd\enddocument{} is expanded.  This will grant us a slightly safer
%   approach that won't depend so much on the internals of
%   \cmd\enddocument.
%
%   This entire patch should work with the previous definition of
%   \cmd\enddocument{} as well (except it cannot be used in the hook),
%   but for now leave previous versions untouched.
%
%   The entire patching will reside in the \texttt{enddocument} hook:
%    \begin{macrocode}
\rvtx@ifformat@geq{2020-10-01}{%
  \AddToHook{enddocument}{\rvtx@enddocument@patch{}}%
}{}
%    \end{macrocode}
%
%   This macro will be executed after \cmd\enddocument{} has expanded,
%   so all its tokens are now exposed.  Here we will assume that
%   \cmd\enddocument{} contains the tokens \verb|\@checkend{document}|
%   and \cmd\endgroup, and use them as delimiters:
%    \begin{macrocode}
\protected\long\def\rvtx@enddocument@patch#1#2\@checkend#3{%
  \begingroup
    \edef\x{\detokenize{#3}}%
    \edef\y{\detokenize{document}}%
  \expandafter\endgroup
  \ifx\x\y
    \expandafter\rvtx@enddocument@patch@end
  \else
    \expandafter\rvtx@enddocument@patch@more
  \fi
    {#1#2}{#3}}
\def\rvtx@enddocument@patch@more#1#2{%
  \rvtx@enddocument@patch{#1\@checkend{#2}}}
%    \end{macrocode}
%
%   When the \verb|\@checkend{document}| is reached, use \cmd\clearpage{}
%   and \cmd\enddocument{} as delimiters for the
%   \verb|<read main .aux and final checks>| part, and save it in
%   \cmd\do@check@aux{}:
%    \begin{macrocode}
\long\def\rvtx@enddocument@patch@end#1#2\clearpage#3\endgroup{%
  \def\do@check@aux{#3\endgroup}%
%    \end{macrocode}
%   Then execute the code consumed in the previous step:
%    \begin{macrocode}
  #1%
  \@checkend{#2}%
%    \end{macrocode}
%   Do \cmd\clear@document{} instead of \cmd\clearpage{} and
%   \cmd\check@aux{} instead of the code grabbed.
%    \begin{macrocode}
  \clear@document
  \check@aux}
\def\check@aux{\do@check@aux}%
%    \end{macrocode}
% \end{macro}
%
% \end{macro}
% \end{macro}
% \end{macro}
%
% \begin{macro}{\clear@document}
% \changes{4.1n}{2009/12/05}{(AO, 569) Use of \classname{hyperref} interferes with column balancing of last page}%
% The procedure \cmd\clear@document\ is responsible for flushing out the last page of the document,
% if not already done. 
% The procedure then executes those procedures that must wait for execution until
% after the last page is shipped out. 
% Clients of \classname{ltxutil}, such as \classname{ltxgrid} and \classname{revtex4}
% will queue these procedures up via \cmd\AfterLastShipout, if it exists, otherwise by doing \cmd\appdef\cmd\clear@document. 
% 
% The command \cmd\Call@AfterLastShipout\ is provided by Heiko Oberdiek's \classname{atveryend} package.
% This package is compatible with \classname{ltxutil}. 
%
% Note on compatibility with \classname{atveryend}: 
% we arrange for \cmd\Call@AfterLastShipout\ to be called from the safety of the output routine,
% thereby ensuring that all of the procedures queued up by that package's
% \cmd\AfterLastShipout\ are executed at the right time.
% We also ensure that \cmd\Call@AfterLastShipout\ has a default definition, in case the package was never loaded. 
% \changes{4.1n}{2009/12/09}{(AO, 569) execute \classname{atveryend}'s \cs{Call@AfterLastShipout} at the proper time}%
%    \begin{macrocode}
\def\clear@document{%
 \clearpage
 \do@output@cclv{%
  \Call@AfterLastShipout
 }%
}%
\appdef\class@documenthook{%
 \providecommand\Call@AfterLastShipout{}%
}%
%    \end{macrocode}
% \end{macro}
%
% \subsection{Class Extensions}%
% \changes{4.1a}{2008/08/12}{Class extension mechanism \cs{class@extension}, \cs{class@extensionfile}, and \cs{class@ext@hook}.}
% The \LaTeX\ procedure \cmd\@onefilewithoptions\ is the vehicle for reading in 
% a \LaTeX\ class or package. 
% The APS RevTeX class implements the use of what are called ``substyles'',
% actually extensions to the class itself. 
% Any document class can do likewise. 
% 
% \begin{macro}{\class@extension}
% \begin{macro}{\class@extensionfile}
% \begin{macro}{\class@ext@hook}
% A procedure similar to \LaTeX's \cmd\@onefilewithoptions, but as an extension to the
% current document class. 
% 
% Read in the given file as if it were a document class file.
% Usage: \cmd\class@extensionfile\ \arg{class} \cmd\@extension,
% where \marg{class} is a file (similar to \file{aps.rtx}) and where \cmd\@extension\ is 
% the file extension for \marg{class}. 
% For instance, to read in  the file \file{aps.rtx}, 
% do \cmd\class@extensionfile\ \texttt{\{aps\}} \cmd\substyle@ext,
% where the latter has been define to expand to \file{.rtx}. 
%
% Features supported include
% passing existing class options on to the class extension,
% \cmd\AtEndOfClass\ processing,
% a stack that restores \cmd\@currname, \cmd\@currext, \cmd\@clsextension, and the \cmd\catcode\ of `@',
% fall-back to a control sequence name (with leading `rtx@') if no file exists.
%
% Note that \cmd\LoadClass\ gives one the ability to write a class that calls in 
% another class as a (sort of) module: this scheme is like \cmd\LoadClass, but
% turned inside out. 
%    \begin{macrocode}
\def\class@extension#1#2{%
 \IfFileExists{#1.#2}{%
  \expandafter\class@extensionfile\csname ver@\@currname.\@currext\endcsname{#1}#2%
 }{%
  \csname rtx@#1\endcsname
 }%
}%
\def\class@extensionfile#1#2#3{%
 \@pass@ptions#3\@unusedoptionlist{#2}%
 \global\let\@unusedoptionlist\@empty
 \expandafter\class@ext@hook\csname#2.#3-h@@k\endcsname#1{#2}#3%
}%
\def\class@ext@hook#1#2#3#4{%
 \@pushfilename@ltx
 \makeatletter
 \let\CurrentOption\@empty
 \@reset@ptions
 \let#1\@empty
 \xdef\@currname{#3}%
 \global\let\@currext#4%
 \global\let\@clsextension\@currext
 \input{#3.#4}%
 \@ifl@ter#4{#3}#2{%
  \class@info{Class extension later than: #2}%
 }{%
  \class@info{Class extension earlier: #2}%
  \@@end
 }%
 #1%
 \let#1\@undefined
 \expandafter\@p@pfilename@ltx\@currnamestack@ltx\@nil
 \@reset@ptions
}%
%    \end{macrocode}
% \end{macro}
% \end{macro}
% \end{macro}
%
% \begin{macro}{\@pushfilename}
% \begin{macro}{\@p@pfilename}
% \changes{4.1a}{2008/08/12}{Class extension mechanism \cs{@pushfilename@ltx} and \cs{@p@pfilename@ltx}.}
% But! \LaTeX\ does not provide for a class extension other than \file{.cls}, therefore we 
% must extend \LaTeX's file name stack with the file extension of a class extension. 
% This way, procedures like 
% \cmd\ProvidesPackage, \cmd\OptionNotUsed, \cmd\ProcessOptions, \cmd\@reset@ptions\ 
% will still work properly. 
%    \begin{macrocode}
\def\@pushfilename@ltx{%
 \xdef\@currnamestack@ltx{%
  {\@currname}%
  {\@currext}%
  {\@clsextension}%
  {\the\catcode`\@}%
  \@currnamestack@ltx
 }%
}%
\def\@p@pfilename@ltx#1#2#3#4#5\@nil{%
 \gdef\@currname{#1}%
 \gdef\@currext{#2}%
 \gdef\@clsextension{#3}%
 \catcode`\@#4\relax
 \gdef\@currnamestack@ltx{#5}%
}%
\global\let\@currnamestack@ltx\@empty
%    \end{macrocode}
% \end{macro}
% \end{macro}
% We carefully patch \LaTeX\ so that the current value of \cmd\@clsextension\ can be
% restored after reading in a class file. 
%
% \subsection{Type Tools}%
%
% \begin{macro}{\flushing}
% Undoes \cmd\centering. Should also undo \cmd\raggedleft\ and \cmd\raggedright.
%    \begin{macrocode}
\def\flushing{%
  \let\\\@normalcr
  \leftskip\z@skip
  \rightskip\z@skip
  \@rightskip\z@skip
  \parfillskip\@flushglue
}%
%    \end{macrocode}
% \end{macro}
%
% \changes{4.1g}{2009/10/07}{(AO, 539) Use of double-backslash in argument of \cs{section} gives error. The \classname{textcase} package is involved.}%
% \begin{macro}{\@centercr}
% The \cmd\@centercr\ command is the replacement for \cmd\@normalcr\ when setting type centered or ragged.
% Normally, the meaning of \cmd\\ is \cmd\@normalcr, which \LaTeX\ defines via \cmd\DeclareRobustCommand.
% In centered or ragged typesetting, the meaning of \cmd\\ is \cmd\@centercr,
% therefore it ought to to be defined via \cmd\DeclareRobustCommand\ (but unfortunately is not).
% The fact that it is not is yet another of \LaTeX's early failures that will never get fixed.
%
% The following exemplar fails under \LaTeX\ version 2005/12/01, package \classname{textcase} 2004/10/07 v0.07:
%    \begin{verbatim}
%\documentclass{article}%
%\usepackage[overload]{textcase}
%\begin{document}
%\centering
%\section{\MakeTextUppercase{Section\\title}}
%Text
%\end{document}
%    \end{verbatim}
% The solution is to promote \cmd\@centercr\ to a robust command, just the same as \cmd\\.
% We do that here without needing to know the meaning of the command.
%    \begin{macrocode}
\expandafter\DeclareRobustCommand\expandafter\@centercr\expandafter{\@centercr}%
%    \end{macrocode}
% \end{macro}
%
%
% \subsection{Display Math}%
%
% \begin{macro}{\eqnarray@LaTeX}
% \begin{macro}{\eqnarray@fleqn@fixed}
% Team \LaTeX\ has stated they will never repair Leslie's broken definition of \env{eqnarray}.
% Let us be bold\dots.
%
% Note on \classname{hyperref} package compatibility: that package overrides
% \cmd\eqnarray\ by wrapping it up in a larger procedure, so its changes
% are compatible with this package's changes.
%
% \changes{4.1a}{2008/01/19}{(AO, 475) I had not properly reproduced the LaTeX macro \cs{eqnarray}.}%
%    \begin{macrocode}
\def\eqnarray@LaTeX{%
   \stepcounter{equation}%
   \def\@currentlabel{\p@equation\theequation}%
   \global\@eqnswtrue
   \m@th
   \global\@eqcnt\z@
   \tabskip\@centering
   \let\\\@eqncr
   $$\everycr{}\halign to\displaywidth\bgroup
       \hskip\@centering$\displaystyle\tabskip\z@skip{##}$\@eqnsel
      &\global\@eqcnt\@ne\hskip \tw@\arraycolsep \hfil${##}$\hfil
      &\global\@eqcnt\tw@ \hskip \tw@\arraycolsep
         $\displaystyle{##}$\hfil\tabskip\@centering
      &\global\@eqcnt\thr@@ \hb@xt@\z@\bgroup\hss##\egroup
         \tabskip\z@skip
      \cr
}
\long\def\eqnarray@fleqn@fixed{%
 \stepcounter{equation}\def\@currentlabel{\p@equation\theequation}%
 \global\@eqnswtrue\m@th\global\@eqcnt\z@
 \tabskip\ltx@mathindent
 \let\\=\@eqncr
 \setlength\abovedisplayskip{\topsep}%
 \ifvmode\addtolength\abovedisplayskip{\partopsep}\fi
 \addtolength\abovedisplayskip{\parskip}%
 \setlength\belowdisplayskip{\abovedisplayskip}%
 \setlength\belowdisplayshortskip{\abovedisplayskip}%
 \setlength\abovedisplayshortskip{\abovedisplayskip}%
 $$%
 \everycr{}%
 \halignt@\linewidth\bgroup
  \hskip\@centering$\displaystyle\tabskip\z@skip{##}$\@eqnsel
  &\global\@eqcnt\@ne
   \hskip\tw@\eqncolsep
   \hfil${{}##{}}$\hfil
  &\global\@eqcnt\tw@
   \hskip\tw@\eqncolsep
   $\displaystyle{##}$\hfil\tabskip\@centering
  &\global\@eqcnt\thr@@\hb@xt@\z@\bgroup\hss##\egroup
   \tabskip\z@skip
  \cr
}%
\@ifx{\eqnarray\eqnarray@LaTeX}{%
 \class@info{Repairing broken LaTeX eqnarray}%
 \let\eqnarray\eqnarray@fleqn@fixed
 \newlength\eqncolsep
 \setlength\eqncolsep\z@
 \let\eqnarray@LaTeX\relax
 \let\eqnarray@fleqn@fixed\relax
}{}%
%    \end{macrocode}
% The macro \cmd\ltx@mathindent\ is assigned to the \cmd\tabskip\ glue just
% before the alignment preamble is expanded, the value therefore applying at the
% left of the first column.
%
% The below value specifies the display math to be set centered, as is common practice.
% Alternatively, \cmd\tabskip\ can be set to a different glue value, accomplishing 
% flush-left display math. 
% 
% Note that the \filename{fleqn.clo} package provides its own meaning for 
% the eqnarray environment, which is also broken. We do not patch that package, however.
% 
% \changes{4.1d}{2009/03/27}{(AO, 511) Compatability with lineno.sty's erroneous way of detecting fleqn.clo}%
% Bug note: The \filename{lineno.sty} package detects \filename{fleqn.clo} 
% by testing whether \cmd\mathindent\ is defined, instead of using correct \LaTeXe\ means.
% Even though our eqnarray environment is modelled after \filename{fleqn.clo},
% we must program defensively here. 
%    \begin{macrocode}
\def\ltx@mathindent{\@centering}%
\def\set@eqnarray@skips{}%
%    \end{macrocode}
% \end{macro}
% \end{macro}
% 
% \changes{4.1g}{2009/10/07}{(AO, 525) Remove phantom paragraph above display math that is given in vertical mode}%
%\begin{macro}{\prep@math}
%\begin{macro}{\prep@math@patch}
% If we are in vertical mode when display math mode is entered (via \verb+$$+), 
% \TeX\ will first enter horizontal mode, then display math mode; this results in a phantom paragraph
% containing a single \cmd\hbox\ consisting of the \cmd\parindent\ box followed by the \cmd\parskipfillskip\ glue. 
% Of course, that \cmd\hbox\ is accompanied by \cmd\parskip\ glue and \cmd\baselineskip\ glue.
%
% The \cmd\prep@math\ procedure removes the \cmd\parindent\ box, thereby (magically) eliminating the phantom paragraph.
% The \cmd\prep@math@patch\ procedure headpatches the \env{equation} and \env{eqnarray} environments
% to accomplish this removal of the phantom paragraph.
%
% Note that there are three remaining ways to enter display math mode that we do not treat:
% the \env{displaymath} environment (equivalent to \cmd\[/\cmd\]), and the primitive the \verb+$$+ markup.
% I refrain from treating the first case because \env{displaymath} already detects the case
% where it is entered from vertical mode: I do not wish to engage in the dubious enterprise of attempting to correct a procedure that is ill conceived from the outset. 
% As to the primitive \verb+$$+, there is no help for users who insist upon employing procedural markup in their documents. 
% in their documents.
%    \begin{macrocode}
\def\prep@math{%
 \@ifvmode{\everypar{{\setbox\z@\lastbox}}}{}%
}%
\def\prep@math@patch{%
 \prepdef\equation{\prep@math}%
 \prepdef\eqnarray{\prep@math}%
}%
%    \end{macrocode}
% A document class may invoke \cmd\prep@math@patch\ at any point it wishes to prevent the appearance
% of the phantom paragraph: it may be a global declaration or a local one.
% \end{macro}
% \end{macro}
%
% We fail to patch \cmd\[, \cmd\equation, however.
%
% \subsection{Footnotes}
%
% \changes{4.0d}{2000/04/10}{(AO, 174) kernel fix}
% 
% \begin{macro}{\footnotemark}
% \begin{macro}{\footnotetest}
% \begin{macro}{\ltx@xfootnote}
% \begin{macro}{\ltx@footmark}
% \begin{macro}{\ltx@foottext}
% \begin{macro}{\ltx@make@current@footnote}
% We repair an error in the \LaTeX\ kernel (see \file{ltfloat.dtx}) involving footnotes.
% The symptom is that the \cmd\footnotemark\ command does not work properly within a \env{minipage} environment.
% The source of the problem is in the way the \cmd\footnotemark\ and \cmd\@xfootnotemark\ procedures are defined:
% they do not share the method, used by \cmd\footnote\ and other procedures, that allows a context switch to 
% change the way footnotes behave within a minipage environment.
% This is a \LaTeX\ bug of long standing; our fix dates to 1987. 
%
% While we are at it, we rewrite both the \cmd\footnote, \cmd\footnotemark\ and \cmd\footnotetext\ procedures,
% achieving a cleaner separation of syntax and semantics. 
% Note that the \cmd\@footnotetext\ procedure is not involved in context switching; 
% \classname{hyperref} will take over that procedure, substituting its own processing around
% its argument and passing this to \cmd\H@@footnotetext. 
% We anticipate this, and do our context switching on \cmd\H@@footnotetext.
%
% The \cmd\@makefnmark\ continues as the method of formatting the footnote mark. 
%
% A note about the context switch mentioned above:
% the \env{minipage} environment executes the following in order to alter the way footnotes
% behave:\begin{verbatim}
%\def\@mpfn{mpfootnote}%
%\def\thempfn{\thempfootnote}%
%\let\@footnotetext\@mpfootnotetext
%\let\@makefnmark\@mpmakefnmark
%\c@mpfootnote\z@\end{verbatim}
% This code changes the counter used in autonumbered footnotes, the choice of footnote marker,
% and the procedure used on the footnote text. Changing the counter is needed because minipage
% footnotes are in their own sequence, and the footnote marker is customarily different within
% a minipage. The procedure that works on the footnote text must be different because the footnotes
% are placed at the bottom of the minipage, not the bottom of the text column.
%
% Note that \LaTeX\ initially defines \cmd\@mpfn\ as \texttt{footnote} and \cmd\thempfn\ as \cmd\thefootnote,
% so we are initially doing general footnotes. 
%
% Any procedure that establishes a minipage-like context (e.g., floats) can do the same as the mimipage context switch 
% illustrated above.
% \changes{4.1a}{2008/06/30}{(AO, 438) Complete rewrite of footnote macros.}
%
% Three user-level command, \cmd\footnote, \cmd\footnotemark, and \cmd\footnotetext\ are defined (see the \LaTeX\ manual for user-level details).
% \begin{macro}{\footnote}
% The first user-level command is \cmd\footnote. 
% A simple way to look at this command is to think of it as
% \cmd\footnotemark\ \oarg{number} \cmd\footnotetext\ \oarg{number} \arg{text},
% where the optional argument is the same in both calls.
% We also define a syntactical helper procedure \cmd\ltx@xfootnote.
% 
% We employ the procedures \cmd\ltx@stp@footproc\ and \cmd\ltx@def@footproc,
% passing in the procedure to execute, in this case \cmd\ltx@footmark,
% which sets the footnote mark.
% In any case, we end on the procedure \cmd\ltx@foottext, which sets the footnote text.
%    \begin{macrocode}
\def\footnote{\@ifnextchar[\ltx@xfootnote\ltx@yfootnote}%
\def\ltx@xfootnote[#1]{%
 \ltx@def@footproc\ltx@footmark[#1]%
 \expandafter\ltx@foottext\expandafter{\the\csname c@\@mpfn\endcsname}%
}%
\def\ltx@yfootnote{%
 \ltx@stp@footproc\ltx@footmark
 \expandafter\ltx@foottext\expandafter{\the\csname c@\@mpfn\endcsname}%
}%
%    \end{macrocode}
% The \cmd\footmark\ user-level command is next. 
% Here we use the procedures \cmd\ltx@stp@footproc\ and \cmd\ltx@def@footproc\ again,
% but unlike \cmd\footnote, we do not set the footnote text.
%    \begin{macrocode}
\def\footnotemark{\@ifnextchar[\ltx@xfootmark\ltx@yfootmark}%
\def\ltx@xfootmark{\ltx@def@footproc\ltx@footmark}%
\def\ltx@yfootmark{\ltx@stp@footproc\ltx@footmark}%
\def\ltx@footmark#1{%
 \leavevmode
 \ifhmode\edef\@x@sf{\the\spacefactor}\nobreak\fi
 \begingroup
  \expandafter\ltx@make@current@footnote\expandafter{\@mpfn}{#1}%
  \expandafter\@argswap@val\expandafter{\Hy@footnote@currentHref}{\hyper@linkstart {link}}%
   \@makefnmark
  \hyper@linkend
 \endgroup
 \ifhmode\spacefactor\@x@sf\fi
 \relax
}%
%    \end{macrocode}
% The third user-level command is \cmd\footnotetext.
% As with \cmd\footnotemark, we use the procedures \cmd\ltx@stp@footproc\ and \cmd\ltx@def@footproc,
% this time passing in the procedure \cmd\ltx@foottext, which sets the footnote text.
%    \begin{macrocode}
\def\footnotetext{\@ifnextchar[\ltx@xfoottext\ltx@yfoottext}%
\def\ltx@xfoottext{\ltx@def@footproc\ltx@foottext}%
\def\ltx@yfoottext{\ltx@stp@footproc\ltx@foottext}%
\long\def\ltx@foottext#1#2{%
 \begingroup
  \expandafter\ltx@make@current@footnote\expandafter{\@mpfn}{#1}%
  \@footnotetext{#2}%
 \endgroup
}%
%    \end{macrocode}
% Here are the definitions of the procedures \cmd\ltx@stp@footproc\ and \cmd\ltx@def@footproc.
% The require argument is the procedure to execute afterwards, and
% \cmd\ltx@def@footproc\ parses a bracket-delimited argument (it is not optional).
% In each case the given procedure is executed with an argument prepared for it:
% the value of the footnote counter.
%    \begin{macrocode}
\def\ltx@def@footproc#1[#2]{%
 \begingroup
   \csname c@\@mpfn\endcsname #2\relax
   \unrestored@protected@xdef\@thefnmark{\thempfn}%
 \expandafter\endgroup
 \expandafter#1%
 \expandafter{\the\csname c@\@mpfn\endcsname}%
}%
\def\ltx@stp@footproc#1{%
 \expandafter\stepcounter\expandafter{\@mpfn}%
 \protected@xdef\@thefnmark{\thempfn}%
 \expandafter#1%
 \expandafter{\the\csname c@\@mpfn\endcsname}%
}%
%    \end{macrocode}
% Here we provide for our good friend \classname{hyperref} 
% to enter in like a bull in a china shop. If it is not loaded,
% we do what it would have done, but gentlier and without hypertext functionality.
%    \begin{macrocode}
\appdef\class@documenthook{%
 \let\footnote@latex\footnote
 \@ifpackageloaded{hyperref}{}{%
  \let\H@@footnotetext\@footnotetext
  \def\@footnotetext{\H@@footnotetext}%
  \let\H@@mpfootnotetext\@mpfootnotetext
  \def\@mpfootnotetext{\H@@mpfootnotetext}%
 }%
}%
%    \end{macrocode}
% In the following, we must use \LaTeX's rococco equipment in the form of \cmd\protected@edef,
% because of the presence of a font switch in the meaning of \cmd\thempfootnote.
% But, really, isn't this a sloppy conflation of semantics and presentation?
%    \begin{macrocode}
\def\ltx@make@current@footnote#1#2{%
  \csname c@#1\endcsname#2\relax
  \protected@edef\Hy@footnote@currentHref{\@currentHref-#1.\csname the#1\endcsname}%
}%
\def\thempfootnote@latex{{\itshape \@alph \c@mpfootnote }}%
\def\ltx@thempfootnote{\@alph\c@mpfootnote}%
\@ifx{\thempfootnote\thempfootnote@latex}{%
 \class@info{Repairing hyperref-unfriendly LaTeX definition of \string\mpfootnote}%
 \let\thempfootnote\ltx@thempfootnote
}{}%
%    \end{macrocode}
%
% Note on \classname{hyperref} compatibility:
% In its ``Automated \LaTeX\ hypertext cross-references'',
% the \classname{hyperref} package alters footnote processing,
% but it does nothing to address the several issues of concern to us.
%
% The \classname{hyperref} package takes over the \cmd\@mpfootnotetext\ and 
% \cmd\@footnotetext\ procedures, wrapping the argument in its own code.
% It also rewrites \cmd\@footnotemark, making it a hyperlink, and
% \cmd\@xfootnotenext, removing from it all hypertext capabilities. 
%
% However, if the \cmd\footnotemark\ command has been supplied with an optional argument,
% \classname{hyperref}'s changes do not apply: it punts in this case.
%
% At the same time, it attempts to turn off its changes during 
% \cmd\maketitle\ processing, destroying one of the capabilities we desire. 
%
% We make ourself \classname{hyperref} savvy:
% we re-implement footnote processing, using \classname{hyperref} capabilities if that 
% package has been loaded. 
%
% Any other package that rewrites \LaTeX's footnote macros will be incompatible
% with this package.
% \changes{4.1a}{2008/06/30}{(AO) Remove code that avoided changes to \cs{@xfootnotemark}}%
%
% Two thoughts about \classname{hyperref}: what for does it define \cmd\realfootnote?
% Apparently even SR himself cannot remember.
%
% Also: a document class that desires high hypertext capabilities might
% well wish to reimplement \cmd\maketitle\ so that footnotes called out from there
% are hypertext links: the \classname{hyperref} package's 
% ``Automated \LaTeX\ hypertext cross-references''
% does not do any of this:
%\begin{quotation}
% But the special footnotes in |\maketitle| are much too hard to deal with properly.
% Let them revert to plain behaviour.
%\end{quotation}
% Note that the document class, in reimplementing \cmd\maketitle, must ensure
% that the \classname{hyperref} package does not clobber its own definition!
%
% \end{macro}
% \end{macro}
% \end{macro}
% \end{macro}
% \end{macro}
% \end{macro}
% \end{macro}
%
% \begin{macro}{\@footnotetext}
% \begin{macro}{\@mpfootnotetext}
% \begin{macro}{\@tpfootnotetext}
% \begin{macro}{\make@footnotetext}
% \begin{macro}{\set@footnotewidth}
% The two procedures \cmd\@footnotetext\ and \cmd\@mpfootnotetext\ share code.
% We make that explicit here.
%
% Note that the procedure calling \cmd\make@footnotetext\ will open a group
% with \cmd\bgroup\ which is then closed by \cmd\minipagefootnote@drop.
%
% Difference from \LaTeX: here we do not set \cmd\floatingpenalty\ to infinity.
% Doing this must date back to a time when \LaTeX\ could not accomodate split insertions (footnotes).
% I cannot think of any other reason to do have done this.
% At any rate, with the \classname{ltxgrid} package, split insertions are properly
% taken care of, so we allow it.
%
% We provide the hook \cmd\set@footnotewidth\
% that sets the footnote on a particular measure.
% Some page grids are such as to set a footnote in a context where \cmd\columnwidth
% is not the right parameter to use for the set width of a footnote.
% In such a case, for the applicable scope, you should define
% \cmd\set@footnotewidth\ to perform this job correctly.
% 
% If we are setting type on multiple page grids, we must still ensure that all footnotes that 
% find their way into the \cmd\footins\ insert register are set on the same width. 
% This implies the need for a document to have an ``overall'' page grid, which determines the set width of all footnotes
% with the exception of minipage footnotes. 
%
% In general, remember that footnotes, like all insertions (including floats), 
% are a step outside of the galley context, and all aspects of insertions need to be
% properly handled, including the set width.
%    \begin{macrocode}
\def\@makefnmark{%
 \hbox{%
  \@textsuperscript{%
   \normalfont\itshape\@thefnmark
  }%
 }%
}%
%    \end{macrocode}
%    \begin{macrocode}
\long\def\@footnotetext{%
 \insert\footins\bgroup
  \make@footnotetext
}%
%    \end{macrocode}
%    \begin{macrocode}
\long\def\@mpfootnotetext{%
 \minipagefootnote@pick
  \make@footnotetext
}%
%    \end{macrocode}
% Procedure \cmd\make@footnotetext\ sets the footnote \verb+#1+ into type, with the proper 
% font, color, leading, width, and label in effect. 
% It also establishes a strut and null glue at the end of the last paragraph of the footnote;
% The strut helps compensate for the lack of \cmd\interlineskip\ glue between \cmd\insert s;
% the glue establishes a feasible \cmd\vsplit\ point between footnotes. 
%
% Note that in the title block (\classname{ltxfront}), the alternative definition, under the name \cmd\frontmatter@footnotetext, is used.
% The only material difference there is the reference to \cmd\frontmatter@makefntext\ instead of \cmd\@makefntext. 
% 
% Dependency note: the \cmd\@makefntext\ procedure is used to further process the footnote text 
% and to execute the \cmd\@makefnmark\ procedure to produce the footnote mark. 
% The definition of the former is customarily found in the document class (hereunder that of \filename{article.cls}), 
% the latter in \filename{latex.ltx}. They are as follows: 
%    \begin{verbatim}
%\newcommand\@makefntext[1]{%
% \parindent 1em\noindent
% \hb@xt@1.8em{\hss\@makefnmark}%
% #1%
%}%
%\def\@makefnmark{%
% \hbox{\@textsuperscript{\normalfont\@thefnmark}}%
%}%
%    \end{verbatim}
% 
% \changes{4.1n}{2010/01/06}{(AO, 572) title block footnotes numbered independently from body footnotes}%
%    \begin{macrocode}
\long\def\make@footnotetext#1{%
  \set@footnotefont
%    \end{macrocode}
% As noted above, we do \emph{not} do \cmd\floatingpenalty\ \cmd\@MM, as in standard \LaTeX. 
%    \begin{macrocode}
  \set@footnotewidth
  \@parboxrestore
  \protected@edef\@currentlabel{%
%    \end{macrocode}
% Note that we employ \cmd\@mpfn\ as a level of redirection for the footnotecounter.
%    \begin{macrocode}
   \csname p@\@mpfn\endcsname\@thefnmark
  }%
  \color@begingroup
   \@makefntext{%
    \rule\z@\footnotesep\ignorespaces#1%
%    \end{macrocode}
% \changes{4.1n}{2010/01/02}{(AO, 571) allow split after last line of footnote}%
% The following strut and glue are for spacing and splitting, as mentioned above. 
%    \begin{macrocode}
    \@finalstrut\strutbox\vadjust{\vskip\z@skip}%
   }%
  \color@endgroup
 \minipagefootnote@drop
}%
%    \end{macrocode}
% \changes{4.1f}{2009/07/07}{(AO, 515) Hook for setting the font of a footnote}
% \cmd\set@footnotefont\ is the procedure for setting the font of a footnote.
% Other aspects of the environment may be set using this hook.
%    \begin{macrocode}
\def\set@footnotefont{%
  \reset@font\footnotesize
  \interlinepenalty\interfootnotelinepenalty
  \splittopskip\footnotesep
  \splitmaxdepth\dp\strutbox
}%
%    \end{macrocode}
% \cmd\set@footnotewidth\ is the procedure for setting the width of a footnote.
% The default page grid, a single, full-width column, sets footnotes on the width of the text. 
% \changes{4.1n}{2010/01/02}{(AO, 571) Interface \cs{set@footnotewidth} for determining the set width of footnotes}%
%    \begin{macrocode}
\def\set@footnotewidth{\set@footnotewidth@one}%
%    \end{macrocode}
% \end{macro}
% \end{macro}
% \end{macro}
% \end{macro}
% \end{macro}
%
%
% \subsection{Floats}
%
% \subsubsection{Usage notes}%
% We extend the \LaTeX\ kernel for three purposes:
% \begin{enumerate}
%
% \item
%  When the \cmd\footnote\ command is used within the
%  scope of a float, we do as \env{minipage} does.
%
% \item
%  We provide a mechanism to write floats out to an external
%  stream for temporary storage (deferred floats).
%
% \item
%  We provide mechanism for placing a float \texttt{here}
%  invariably, that is, floats are unfloated.
%  This mechanism is used to read the external stream mentioned above.
%
% \end{enumerate}
%
% To use these mechanisms, the document class should 
% define a float, say, \env{figure} as per usual, and in addition:
% \begin{enumerate}
%
% \item
%  Optionally define an alternative, say \env{figure@write} as follows:
%    \begin{verbatim}\newenvironment{figure@write}{%
% \write@float{figure}%
%}{%
% \endwrite@float
%}\end{verbatim}
% That is, the alternative environment executes \cmd\write@float\
% instead of \cmd\@float.
% Note that this step is not needed if the float environment
% is defined in the simple way of \file{classes.dtx}.
% However, an environment like \env{longtable} will require it.
%
% \item
%  Install into \cmd\AtBeginDocument\ a call to \cmd\do@if@floats,
%  with the float name and an appropriate file extension as its arguments.
%    \begin{verbatim}\appdef\class@documenthook{\do@if@floats{figure}{.fgx}}\end{verbatim}
%  
% \item
%  Optionally define a text entity \cmd\figuresname\ that will
%  be the text of the head that is set over the 
%  deferred floats.
%  If not defined, there will be no head.
%  
% \item
%  Optionally define a user-level command to allow
%  the document to determine where the figures are printed out
%  (default is to print at end of document). E.g.,
%    \begin{verbatim}\newcommand\printfigures{\print@float{figure}}\end{verbatim}
% \item
%  Install into \cmd\appdef\cmd\class@enddocumenthook\ a call to \cmd\printfigures,
%  or, if the latter is not defined, as follows:
%    \begin{verbatim}\appdef\class@enddocumenthook{\print@float{figure}}\end{verbatim}
%  Note that installing this command into \cmd\AtBeginDocument
%  is best done earlier than calls that assume the last page of
%  the document is at hand.
%  
% \end{enumerate} 
%
% \subsubsection{Robustifying fragile commands}%
% Certain of \LaTeX's  commands cannot be written out to a file or appear within a \cmd\mark\ command argument
% because they do calculations during expansion.
% We provide for a little help, but without changing the meanings of these commands.
%
% \begin{macro}{\addtocontents}
% \begin{macro}{\robustify@contents}
%
% \changes{4.1a}{2008/06/30}{(AO) Make \cs{addtocontents} a \cs{long} \cs{def}; gobble up \cs{footnote}}%
%    \begin{macrocode}
\def\robustify@contents{%
 \let \label \@gobble
 \let \index \@gobble
 \let \glossary \@gobble
 \let\footnote \@gobble
 \def\({\string\(}%
 \def\){\string\)}%
 \def\\{\string\\}%
}%
\long\def\addtocontents#1#2{%
 \protected@write\@auxout{\robustify@contents}{\string \@writefile {#1}{#2}}%
}%
%    \end{macrocode}
%
% \end{macro}
% \end{macro}
%
% \subsubsection{Preparing for the \classname{hyperref package}}%
%
% \begin{macro}{\addcontentsline}
% \begin{macro}{\label}
% \begin{macro}{\ltx@contentsline}
% \changes{4.1b}{2008/08/12}{Refine toc processing: provide default.}%
% The \classname{hyperref} package assumes that the \cmd\contentsline\ command will be given four arguments.
% Therefore it cannot successfully process a \filename{.toc} file that had been written by standard \LaTeX.
% We fix things up by always writing that fourth argument and by supplying a \cmd\contentsline\ command that
% can read them.
%
% We also give the \cmd\newlabel\ command's second argument five tokens.
%
% Finally, we wrap \LaTeX's \cmd\contentsline\ command with code to detect the case
% where the expected procedure is not defined, and we give it a syntax with no semantics.
%
% We switch over to this new definition only after \classname{hyperref} has loaded.
%    \begin{macrocode}
\def\addcontentsline#1#2#3{%
 \addtocontents{#1}{%
  \protect\contentsline{#2}{#3}{\thepage}{}%
 }%
}%
\def\label#1{%
 \@bsphack
  \protected@write\@auxout{}{%
   \string\newlabel{#1}{{\@currentlabel}{\thepage}{}{}{}}%
  }%
 \@esphack
}%
\def\ltx@contentsline#1{%
 \expandafter\@ifnotrelax\csname l@#1\endcsname{}{%
  \expandafter\let\csname l@#1\endcsname\@gobbletwo
 }%
 \contentsline@latex{#1}%
}%
\appdef\document@inithook{%
 \let\contentsline@latex\contentsline
 \let\contentsline\ltx@contentsline
}%
%    \end{macrocode}
%
% \end{macro}
% \end{macro}
% \end{macro}
%
% \subsubsection{Footnotes within floats, unfloating floats, float font}%
%
% \begin{macro}{\caption}
% DPC: Er a bit of a hack, but seems best way of supporting normal
% \LaTeX\ syntax at this point: If a caption is used below a table,
% then put out the footnotes before the caption.
% \changes{4.0b}{1999/06/20}{Support the hack with \cs{prepdef}, and delay until \cs{AtBeginDocument} time, since \classname{hyperref} clobbers \cs{caption}.}
%    \begin{macrocode}
\appdef\class@documenthook{%
 \prepdef\caption{\minipagefootnote@here}%
}%
%    \end{macrocode}
%
% Note on \classname{hyperref} compatibility:
% this change to the \cmd\caption\ command is compatible with the
% ``Automated \LaTeX\ hypertext cross-references'' patches of that package.
%
% All the same, I think Sebastian's changes to \cmd\caption\ and \cmd\@caption\
% could bear with some improvement.
% The following implementation requires knowing only the pattern part of the 
% \cmd\@caption\ macro:
%\begin{verbatim}
%\def\caption{%
%  \H@refstepcounter\@captype
%  \hyper@makecurrent{\@captype}%
%  \@dblarg{\H@caption\@captype}%
%}%
%\def\H@caption#1[#2]#3{%
% \@caption{#1}[#2]{%
%  \ifHy@nesting
%   \hyper@@anchor{\@currentHref}{#3}%
%  \else
%   \hyper@@anchor{\@currentHref}{\relax}#3%
%  \fi
% }%
%}\end{verbatim}
%
% \end{macro}
%
% \begin{macro}{\minipagefootnote@init}
% \begin{macro}{\minipagefootnote@here}
% \begin{macro}{\minipagefootnote@foot}
% \begin{macro}{\minipagefootnote@pick}
% \begin{macro}{\minipagefootnote@drop}
% Procedure to deal with footnotes accumulated within a minipage environment.
% These procedures encapsulate all uses of the \cmd\@mpfootins\ box.
% 
%
% Note: \cmd\minipagefootnote@here\ must \emph{not} be executed within the MVL!
%    \begin{macrocode}
\def\minipagefootnote@init{%
 \setbox\@mpfootins\box\voidb@x
}%
\def\minipagefootnote@pick{%
 \global\setbox\@mpfootins\vbox\bgroup
  \unvbox\@mpfootins
}%
\def\minipagefootnote@drop{%
 \egroup
}%
\def\minipagefootnote@here{%
    \par
    \@ifvoid\@mpfootins{}{%
      \vskip\skip\@mpfootins
      \fullinterlineskip
      \@ifinner{%
       \vtop{\unvcopy\@mpfootins}%
       {\setbox\z@\lastbox}%
      }{}%
      \unvbox\@mpfootins
    }%
}%
\def\minipagefootnote@foot{%
 \@ifvoid\@mpfootins{}{%
  \insert\footins\bgroup\unvbox\@mpfootins\egroup
 }%
}%
\def\endminipage{%
    \par
    \unskip
    \minipagefootnote@here
    \@minipagefalse   %% added 24 May 89
  \color@endgroup
  \egroup
  \expandafter\@iiiparbox\@mpargs{\unvbox\@tempboxa}%
}%
%    \end{macrocode}
% \end{macro}
% \end{macro}
% \end{macro}
% \end{macro}
% \end{macro}
%
% \begin{macro}{\floats@sw}
% \changes{4.1b}{2008/08/12}{Default assignment of \cs{float@sw} now, not at \cs{AtBeginDocument} time.}%
% The Boolean \cmd\floats@sw\ signifies that floats are to be floated;
% if false, that floats are to be deferred to the end of the document.
% Note that the assignment of this Boolean is to be overridden by 
% the document class in response to user-selected options.
%    \begin{macrocode}
\@booleantrue\floats@sw
%    \end{macrocode}
% \end{macro}
%
% \begin{macro}{\@xfloat}
% \begin{macro}{\@mpmakefntext}
% The float start-code is redefined to set up footnotes in the style of minipage.
% Also, the \cmd\floats@sw\ Boolean informs us that floats are to be 
% all placed \texttt{here}.
% Note that, to protect against the Boolean being undefined
% at this late hour, we default it globally to true.
%
% \changes{4.0b}{1999/06/20}{AO: Removed superfluous \cs{def}s, changed to using \cs{floats@sw} as the flag. Also stopped using DPC's \cs{if@twocolumn} flag: using \cs{floats@sw} instead. Also added \cs{par}\cs{vskip}\cs{z@skip} after the \cs{minipagefootnotes} so that the float box would have zero depth like the kernel one. }
% \changes{4.0c}{1999/11/13}{(AO, 110) Install hooks for endfloats processing}
% \changes{4.0d}{2000/04/10}{(AO, 127) Floats placed [h] to allow page breaks}
% \changes{4.0d}{2000/05/19}{(AO, 224) Hyperref compatibility.}
% \changes{4.0e}{2000/11/16}{(AO, 221) Remove samepage command from @xfloat@prep: If the float can break over pages, we want better control.}
% \changes{4.1a}{2008/07/07}{\cs{@xfloat@prep} calls \cs{ltx@footnote@pop} to restore the original \cs{ltx@footmark} and \cs{ltx@foottext} procedures, in case footnote processing has switched.}
% \changes{4.1b}{2008/08/12}{No need to protect against undefined \cs{float@sw}}
%
%    \begin{macrocode}
\let\@xfloat@LaTeX\@xfloat
\def\@xfloat#1[#2]{%
  \@xfloat@prep
  \@nameuse{fp@proc@#2}%
  \floats@sw{\@xfloat@LaTeX{#1}[#2]}{\@xfloat@anchored{#1}[]}%
}%
\def\@xfloat@prep{%
  \ltx@footnote@pop
  \def\@mpfn{mpfootnote}%
  \def\thempfn{\thempfootnote}%
  \c@mpfootnote\z@
  \let\H@@footnotetext\H@@mpfootnotetext
}%
\let\ltx@footnote@pop\@empty
\def\@xfloat@anchored#1[#2]{%
 \def\@captype{#1}%
 \begin@float@pagebreak
  \let\end@float\end@float@anchored
  \let\end@dblfloat\end@float@anchored
        \hsize\columnwidth
        \@parboxrestore
        \@floatboxreset
  \minipagefootnote@init
}%
\def\end@float@anchored{%
  \minipagefootnote@here
  \par\vskip\z@skip
 \par
 \end@float@pagebreak
}%
\def\begin@float@pagebreak{\par\addvspace\intextsep}%
\def\end@float@pagebreak{\par\addvspace\intextsep}%
\def\@mpmakefntext#1{%
 \parindent=1em
 \noindent
 \hb@xt@1em{\hss\@makefnmark}%
 #1%
}%
%    \end{macrocode}
% \end{macro}
% \end{macro}
%
%
% \subsubsection{Writing floats out to a file}%
%
% \begin{macro}{\do@if@floats}
% \changes{4.1b}{2008/08/12}{No need to protect against undefined \cs{float@sw}}
% The procedure \cmd\do@if@floats\ should be executed at
% \cmd\class@documenthook\ time: it arranges to write out
% the floats of the given class to a temporary file, to be 
% read back later (deferred floats),
% given that \cmd\floats@sw\ is false.
% Note that, to protect against the Boolean being undefined
% at this late hour, we default it globally to true.
%    \begin{macrocode}
\def\do@if@floats#1#2{%
 \floats@sw{}{%
%    \end{macrocode}
% Open the stream to save out the document's floats of this class.
%    \begin{macrocode}
  \expandafter\newwrite
              \csname#1write\endcsname
  \expandafter\def
              \csname#1@stream\endcsname{\jobname#2}%
  \expandafter\immediate
  \expandafter\openout
              \csname#1write\endcsname
              \csname#1@stream\endcsname\relax
%    \end{macrocode}
% Swap environments.
% If the class writer has defined, e.g., \env{figure@write},
% then we use this as the procedure to execute for writing
% the float out to the external stream.
% Otherwise, the replacement of \cmd\@float\ by \cmd\write@float\
% should do the right thing for float environments defined
% in the simple way of \classname{classes.dtx}.
%    \begin{macrocode}
  \@ifxundefined\@float@LaTeX{%
   \let\@float@LaTeX\@float
   \let\@dblfloat@LaTeX\@dblfloat
   \let\@float\write@float
   \let\@dblfloat\write@floats
  }{}%
  \let@environment{#1@float}{#1}%
  \let@environment{#1@floats}{#1*}%
  \@ifxundefined@cs{#1@write}{}{%
   \let@environment{#1}{#1@write}%
  }%
 }%
}%
%    \end{macrocode}
% \end{macro}
%
% \begin{macro}{\print@float}
% The procedure \cmd\print@float\ prints out the
% deferred floats.
% \changes{4.0b}{1999/06/20}{only execute if there really were floats of the given type}
% \changes{4.0c}{1999/11/13}{*-form mandates pagebreak at each float; only print section head if there is something there.}
% \changes{4.0d}{2000/05/23}{Allow things to break over pages by setting array@default.}
% \changes{4.1b}{2008/08/12}{If class option \classoption{lengthcheck} is in effect, log the height of this float class.}
%
% Here, we make use of the \cmd\floats@sw\ Boolean to select
% the non-floating type of processing.
%    \begin{macrocode}
\def\triggerpar{\leavevmode\@@par}%
\def\oneapage{\def\begin@float@pagebreak{\newpage}\def\end@float@pagebreak{\newpage}}%
\def\print@float#1#2{%
 \lengthcheck@sw{%
  \total@float{#1}%
 }{}%
 \@ifxundefined@cs{#1write}{}{%
  \begingroup
   \@booleanfalse\floats@sw
   #2%
   \raggedbottom
   \def\array@default{v}% floats must 
   \let\@float\@float@LaTeX
   \let\@dblfloat\@dblfloat@LaTeX
   \let\trigger@float@par\triggerpar
   \let@environment{#1}{#1@float}%
   \let@environment{#1*}{#1@floats}%
   \expandafter\prepdef\csname#1\endcsname{\trigger@float@par}%
   \expandafter\prepdef\csname#1*\endcsname{\trigger@float@par}%
   \@namedef{fps@#1}{h!}%
   \expandafter\immediate
   \expandafter\closeout
               \csname#1write\endcsname
   \everypar{%
    \global\let\trigger@float@par\relax
    \global\everypar{}\setbox\z@\lastbox
    \@ifxundefined@cs{#1sname}{}{%
     \begin@float@pagebreak
     \expandafter\section
     \expandafter*%
     \expandafter{%
                  \csname#1sname\endcsname
                 }%
    }%
   }%
   \input{\csname#1@stream\endcsname}%
  \endgroup
  \global\expandafter\let\csname#1write\endcsname\relax
 }%
}%
%    \end{macrocode}
% \end{macro}
%
% \begin{macro}{\tally@float}
% \begin{macro}{\total@float}
% \changes{4.1b}{2008/08/12}{Tally and log the height of a float class}
% If we are tallying column inches, \cmd\tally@float\ tallies a contribution to \cmd\ftype@\ \cmd\@captype,
% depending upon the width of \cmd\@currbox.
% In effect, each float class is tallied in two sections, one for narrow, one for wide floats.
% 
% If statistics are wanted, \cmd\total@float\ logs the tally for the given float class.
% \changes{4.1f}{2009/07/10}{(AO, 518) Tally register overflow when locument is long}
% The quantity \cmd\@twopowerfourteen\ is $2^{14}$, \cmd\@twopowertwo\ is $2^{2}$.
%    \begin{macrocode}
\chardef\@xvi=16\relax
\mathchardef\@twopowerfourteen="4000
\mathchardef\@twopowertwo="4
\def\tally@float#1{%
 \begingroup
%    \end{macrocode}
% We strip all but the least significant 5 bits from \cmd\count\ \cmd\@currbox,
% and put them into \cmd\@tempcnta. We then subtract 16 from \cmd\count\ \cmd\@currbox
% (unless this would make it negative), effectively reversing the process
% carried out in \cmd\@float.
%    \begin{macrocode}
  \@tempcnta\count\@currbox
  \divide\@tempcnta\@xxxii
  \multiply\@tempcnta\@xxxii
  \advance\count\@currbox-\@tempcnta
  \divide\@tempcnta\@xxxii
  \@ifnum{\count\@currbox>\@xvi}{%
   \advance\count\@currbox-\@xvi\@booleantrue\@temp@sw
  }{%
   \@booleanfalse\@temp@sw
  }%
%    \end{macrocode}
% If so desired, we log the characteristics of this float object:
% float class and float placement parameters, height, depth, and width.
%    \begin{macrocode}
  \show@box@size@sw{%
   \class@info{Float #1
    (\the\@tempcnta)[\@temp@sw{16+}{}\the\count\@currbox]^^J%
    (\the\ht\@currbox+\the\dp\@currbox)X\the\wd\@currbox
   }%
  }{}%
 \endgroup
%    \end{macrocode}
% Here we tally the height of this float object.
%    \begin{macrocode}
 \expandafter\let
 \expandafter\@tempa
             \csname fbox@\csname ftype@#1\endcsname\endcsname
 \@ifnotrelax\@tempa{%
  \@ifhbox\@tempa{%
   \setbox\@tempboxa\vbox{\unvcopy\@currbox\hrule}%
   \dimen@\ht\@tempboxa
   \divide\dimen@\@twopowerfourteen
   \@ifdim{\wd\@tempboxa<\textwidth}{%
    \advance\dimen@\ht\@tempa
    \global\ht\@tempa\dimen@
   }{%
    \advance\dimen@\dp\@tempa
    \global\dp\@tempa\dimen@
   }%
  }{}%
 }{}%
}%
\def\total@float#1{%
 \expandafter\let
 \expandafter\@tempa
             \csname fbox@\csname ftype@#1\endcsname\endcsname
 \@ifnotrelax\@tempa{%
  \@ifhbox\@tempa{%
   \@tempdima\the\ht\@tempa\divide\@tempdima\@twopowertwo\@tempcnta\@tempdima
   \@tempdimb\the\dp\@tempa\divide\@tempdimb\@twopowertwo\@tempcntb\@tempdimb
   \class@info{Total #1: Column(\the\@tempcnta pt), Page(\the\@tempcnta pt)}%
  }{}%
 }{}%
}%
%    \end{macrocode}
% \end{macro}
% \end{macro}
%
% \begin{macro}{\write@float}
% \begin{macro}{\write@floats}
% \begin{macro}{\write@@float}
% Handles the case where the name of the float is the same as
% that of the stream. Note that
% \env{longtable} does \emph{not} fit this case.
% Note also: \cmd\write@float\ is \emph{not} a user-level environment,
% therefore it is properly not defined with \cmd\newenvironment.
%    \begin{macrocode}
\def\write@float#1{\write@@float{#1}{#1}}%
\def\endwrite@float{\@Esphack}%
\def\write@floats#1{\write@@float{#1*}{#1}}%
\def\endwrite@floats{\@Esphack}%
%    \end{macrocode}
% \end{macro}
%
% \begin{macro}{\write@@float}
% \changes{4.0b}{1999/06/20}{AO: Fixed spurious \texttt{CR} and (return) characters in output file. Also, if the document did not have the \cs{end}\texttt{figure} on a line of its own, the macro wouldn't work. Fixed.}
%    \begin{macrocode}
\def\write@@float#1#2{%
  \ifhmode
     \@bsphack
  \fi
  \chardef\@tempc\csname#2write\endcsname
  \toks@{\begin{#1}}%
  \def\@tempb{#1}%
  \expandafter\let\csname end#1\endcsname\endwrite@float
  \catcode`\^^M\active
  \@makeother\{\@makeother\}\@makeother\%
  \write@floatline
}%
%    \end{macrocode}
% \end{macro}
% \end{macro}
% \end{macro}
%
% \begin{macro}{\write@floatline}
% \begin{macro}{\@write@floatline}
% \begin{macro}{\float@end@tag}
% The procedure \cmd\write@floatline\ only parses; 
% it passes its result to \cmd\@write@floatline, which
% writes the line to output, then tests the line
% for the \cmd\end\arg{float} tokens with 
% aid of the \cmd\float@end@tag\ procedure.
%    \begin{macrocode}
\begingroup
 \catcode`\[\the\catcode`\{\catcode`\]\the\catcode`\}\@makeother\{\@makeother\}%
 \gdef\float@end@tag#1\end{#2}#3\@nul[%
  \def\@tempa[#2]%
  \@ifx[\@tempa\@tempb][\end[#2]][\write@floatline]%
 ]%
 \obeylines%
 \gdef\write@floatline#1^^M[%
  \begingroup%
   \newlinechar`\^^M%
   \toks@\expandafter[\the\toks@#1]\immediate\write\@tempc[\the\toks@]%
  \endgroup%
  \toks@[]%
  \float@end@tag#1\end{}\@nul%
 ]%
\endgroup
%    \end{macrocode}
% \end{macro}
% \end{macro}
% \end{macro}
%
%
%
% \subsection{Counters}
% The following definitions override those of the \LaTeX\ kernel,
% providing for a greater range of inputs.
%    \begin{macrocode}
\def\@alph#1{\ifcase#1\or a\or b\or c\or d\else\@ialph{#1}\fi}
%    \end{macrocode}
%
%    \begin{macrocode}
\def\@ialph#1{\ifcase#1\or \or \or \or \or e\or f\or g\or h\or i\or j\or
  k\or l\or m\or n\or o\or p\or q\or r\or s\or t\or u\or v\or w\or x\or
  y\or z\or aa\or bb\or cc\or dd\or ee\or ff\or gg\or hh\or ii\or jj\or
  kk\or ll\or mm\or nn\or oo\or pp\or qq\or rr\or ss\or tt\or uu\or
  vv\or ww\or xx\or yy\or zz\else\@ctrerr\fi}
%    \end{macrocode}
%
%
% \subsection{Customization of Sections}%
%
% Patch the standard \LaTeX\ sectioning procedure to:
%\begin{itemize}
%\item
% Allow a sectioning command to trigger the title page, or more generally
% to recognize that it is the first object in the document,
% so we headpatch \cmd\@startsection.
%
%\item
% Allow a tail command in |#6| to uppercase the title, so we retain
% DPC's braces.
%
%\item
% Allow each type of sectioning command to format its number differently,
% so we generalize \cmd\@seccntformat.
%
%\item
% Allow each type of sectioning command to format its argument differently,
% so we generalize \cmd\@hangfrom.
%
%\item
% Allow the starred form of the command to
% mark (the running head) and
% make an entry in the TOC,
% so we put \cmd\@ssect\ on the same footing as \cmd\@sect.
%
% Note that the tokens passed to the TOC now are \emph{not}
% the optional argument of the command, but the required.
% This means that the user can no longer use the former
% to put variant content in to the TOC as the Manual says.
%
% Instead, the optional argument is used to put an alternative
% title into the running headers, a better choice.
%
%\end{itemize}
%
% \begin{macro}{\@startsection}
% Patch a head hook into the basic sectioning command.
% Treat \cmd\@sect\ and \cmd\@ssect\ on an equal footing:
% now their pattern parts are identical.
%    \begin{macrocode}
\def\@startsection#1#2#3#4#5#6{%
 \@startsection@hook
 \if@noskipsec \leavevmode \fi
 \par
 \@tempskipa #4\relax
 \@afterindenttrue
 \ifdim \@tempskipa <\z@
  \@tempskipa -\@tempskipa \@afterindentfalse
 \fi
 \if@nobreak
  \everypar{}%
 \else
  \addpenalty\@secpenalty\addvspace\@tempskipa
 \fi
 \@ifstar
  {\@dblarg{\@ssect@ltx{#1}{#2}{#3}{#4}{#5}{#6}}}%
  {\@dblarg{\@sect@ltx {#1}{#2}{#3}{#4}{#5}{#6}}}%
}%
\def\@startsection@hook{}%
%    \end{macrocode}
% \end{macro}
%
% \begin{macro}{\@sect}
% When defining \cmd\@svsec, do not expand \cmd\@seccntformat.
% Put brace characters back where they were before David Carlisle got at them
% (i.e., as if \cmd\@hangfrom\ had two arguments).
% Protect the mark mechanism from an undefined meaning.
% Pass |#8| to the TOC instead of |#7|.
% Remove \cmd\relax\ from the replacement part of \cmd\@svsec.
%
% The procedure \cmd\@hangfrom\ and \cmd\@runin@to\ can
% be used to process the argument of the head.
% The head can define, e.g., \cmd\@hangfrom@section, to
% do its own processing.
%
% In using \cmd\H@refstepcounter\ in place of \cmd\refstepcounter\ we rely on
% either loading before any package that patches the latter, or 
% the convention that the former is the original \LaTeX\ procedure.
%
%    \begin{macrocode}
\class@info{Repairing broken LateX \string\@sect}%
\def\@sect@ltx#1#2#3#4#5#6[#7]#8{%
  \@ifnum{#2>\c@secnumdepth}{%
    \def\H@svsec{\phantomsection}%
    \let\@svsec\@empty
  }{%
    \H@refstepcounter{#1}%
    \def\H@svsec{%
     \phantomsection
    }%
    \protected@edef\@svsec{{#1}}%
    \@ifundefined{@#1cntformat}{%
     \prepdef\@svsec\@seccntformat
    }{%
     \expandafter\prepdef
     \expandafter\@svsec
                 \csname @#1cntformat\endcsname
    }%
  }%
  \@tempskipa #5\relax
  \@ifdim{\@tempskipa>\z@}{%
    \begingroup
      \interlinepenalty \@M
      #6{%
       \@ifundefined{@hangfrom@#1}{\@hang@from}{\csname @hangfrom@#1\endcsname}%
       {\hskip#3\relax\H@svsec}{\@svsec}{#8}%
      }%
      \@@par
    \endgroup
    \@ifundefined{#1mark}{\@gobble}{\csname #1mark\endcsname}{#7}%
    \addcontentsline{toc}{#1}{%
      \@ifnum{#2>\c@secnumdepth}{%
       \protect\numberline{}%
      }{%
       \protect\numberline{\csname the#1\endcsname}%
      }%
      #8}%
  }{%
    \def\@svsechd{%
      #6{%
       \@ifundefined{@runin@to@#1}{\@runin@to}{\csname @runin@to@#1\endcsname}%
       {\hskip#3\relax\H@svsec}{\@svsec}{#8}%
      }%
      \@ifundefined{#1mark}{\@gobble}{\csname #1mark\endcsname}{#7}%
      \addcontentsline{toc}{#1}{%
        \@ifnum{#2>\c@secnumdepth}{%
         \protect\numberline{}%
        }{%
         \protect\numberline{\csname the#1\endcsname}%
        }%
        #8}%
    }%
  }%
  \@xsect{#5}%
}%
\def\@hang@from#1#2#3{\@hangfrom{#1#2}#3}%
\def\@runin@to #1#2#3{#1#2#3}%
%    \end{macrocode}
% \end{macro}
%
% \begin{macro}{\@ssect}
% Put brace characters back where they were before David Carlisle got at them
% (as if \cmd\@hangfrom\ has two arguments).
% Possibly set a mark.
% Make a TOC entry.
%
% Note that, for compatibility with the \classname{hyperref} package, we
% need to provide the interface required by that package
% (actually required by \file{pdfmark.def} and \file{nameref.sty}),
% namely
% the definition of \cmd\@currentlabelname\ (but now removed),
% the insertion of the procedure \cmd\Sectionformat\ (but why is this needed?), and
% the call to \cmd\phantomsection\ (which must precede the call to \cmd\addcontentsline).
% We also have to sidestep the patch to \cmd\@ssect\ in that same file, therefore
% we use a different control sequence name in the call from \cmd\@startsection.
% \changes{4.0c}{1999/11/13}{(AO, 116) Hyperref compatibility}
% \changes{4.0f}{2001/07/13}{(AO, 404) Hyperref compatibility}
%    \begin{macrocode}
\def\@ssect@ltx#1#2#3#4#5#6[#7]#8{%
%    \end{macrocode}
% Removed |\def\@currentlabelname{#8}|
%    \begin{macrocode}
  \def\H@svsec{\phantomsection}%
  \@tempskipa #5\relax
  \@ifdim{\@tempskipa>\z@}{%
    \begingroup
      \interlinepenalty \@M
      #6{%
       \@ifundefined{@hangfroms@#1}{\@hang@froms}{\csname @hangfroms@#1\endcsname}%
%    \end{macrocode}
% Removed |{\hskip#3\relax\H@svsec}{\Sectionformat{#8}{#1}}|
%    \begin{macrocode}
       {\hskip#3\relax\H@svsec}{#8}%
      }%
      \@@par
    \endgroup
    \@ifundefined{#1smark}{\@gobble}{\csname #1smark\endcsname}{#7}%
    \addcontentsline{toc}{#1}{\protect\numberline{}#8}%
  }{%
    \def\@svsechd{%
      #6{%
       \@ifundefined{@runin@tos@#1}{\@runin@tos}{\csname @runin@tos@#1\endcsname}%
%    \end{macrocode}
% Removed |{\hskip#3\relax\H@svsec}{\Sectionformat{#8}{#1}}|
%    \begin{macrocode}
       {\hskip#3\relax\H@svsec}{#8}%
      }%
      \@ifundefined{#1smark}{\@gobble}{\csname #1smark\endcsname}{#7}%
      \addcontentsline{toc}{#1}{\protect\numberline{}#8}%
    }%
  }%
  \@xsect{#5}%
}%
\def\@hang@froms#1#2{#1#2}%
\def\@runin@tos #1#2{#1#2}%
%    \end{macrocode}
% \end{macro}
%
% \begin{macro}{\init@hyperref}
% \changes{4.1b}{2008/08/12}{Acquire \classname{hyperref} savoire}
% Document classes that incorporate this package will be \classname{hyperref}-savvy.
% (To accomplish this, we ensure that \cmd\hyperanchor\ and \cmd\hyper@last\ are both defined.)
% Being \classname{hyperref}-savvy levels some requirements on us, but the benefits are many.
%
% One is that the TOC will not get amnesia and require a full set of three typesetting runs before its formatting is stable.
% Instead, only two runs are required: the first updates the auxiliary file, the second the TOC.
% However, the formatting of the document does not change.
%
% Another aspect of being \classname{hyperref}-savvy is that the syntax of commands in the \file{.aux} file will not change
% if \classname{hyperref} is turned on or off. 
%
% Note that \cmd\hyper@anchorstart\ and \cmd\hyper@anchorend\ constitute the programming interface
% for a hypertext anchor (the target of a hypertext link); \cmd\hyper@linkstart\ and \cmd\hyper@linkend\
% are the interface for a hypertext link.
%    \begin{macrocode}
\def\init@hyperref{%
 \providecommand\phantomsection{}%
 \providecommand\hyper@makecurrent[1]{}%
 \providecommand\Hy@raisedlink[1]{}%
 \providecommand\hyper@anchorstart[1]{}%
 \providecommand\hyper@anchorend{}%
 \providecommand\hyper@linkstart[2]{}%
 \providecommand\hyper@linkend{}%
 \providecommand\@currentHref{}%
}%
\let\H@refstepcounter\refstepcounter
\appdef\document@inithook{%
 \init@hyperref
}%
%    \end{macrocode}
% \end{macro}
%
% \begin{macro}{\sec@upcase}
% Upper case for sections (optional upper case items).  These are
% created so that some headings can be toggled between mixed case and
% upper case readily.
% Headings that might be changed can be wrapped in the style file in
% \cmd\sec@upcase\arg{text} constructs;
% the expansion of \cmd\sec@upcase\ is
% controlled here.  It is \cmd\relax\ by default (mixed case heads), and
% can easily be changed to \cmd\uppercase\ if desired.
% If mixed-case headings are wanted by the editor, authors {\em must}
% supply mixed case text, although this is what authors should be doing
% anyway.
% (Mixed can be converted to upper,
% but the reverse transformation cannot be automated.)
%
% The following setting gives the \LaTeX\ default.
%    \begin{macrocode}
\def\sec@upcase#1{\relax{#1}}%
%    \end{macrocode}
% \end{macro}
%
%
% \subsection{Patch the \env{tabular} and \env{array} Environments}
%
% \begin{macro}{\endtabular}
% \begin{macro}{\endarray}
% \changes{4.1b}{2008/08/12}{Patch the array package even later: after all package patches go in.}
% We headpatch the begin processing and tailpatch the end processing
% of the \env{tabular} and \env{array} environments.
% A document class can define these hooks as needed.
%
% We proceed with care to make further patches to 
% support tabulars that break over pages.
% Our patches will not necessarily be effective for 
% other packages that replace the \LaTeX\ \env{array} and \env{tabular}
% environments. I know of none that do so.
%    \begin{macrocode}
\appdef\document@inithook{%
 \@ifpackageloaded{array}{\switch@array}{\switch@tabular}%
 \prepdef\endtabular{\endtabular@hook}%
 \@provide\endtabular@hook{}%
 \prepdef\endarray{\endarray@hook}%
 \@provide\endarray@hook{}%
 \providecommand\array@hook{}%
%    \end{macrocode}
% Install, effectively, a head patch to \cmd\tabular.
% In order to avoid interference from, e.g., the \classname{array} package,
% we must perform this patch only \emph{after} packages load.
% \changes{4.0c}{1999/11/13}{(AO, 130) Interference from array package}
%    \begin{macrocode}
 \prepdef\@tabular{\tabular@hook}%
 \@provide\tabular@hook{}%
}%
%    \end{macrocode}
% \end{macro}
% \end{macro}
%
% \begin{macro}{\switch@tabular}
% \begin{macro}{\switch@array}
% The two procedures \cmd\switch@tabular\ and \cmd\switch@array\ 
% apply needed patches to the various tabular procedures,
% the former applying to the \LaTeX\ kernel, the latter to the 
% required \classname{array} package (and to the number of other
% required packages that load it).
%
%    \begin{macrocode}
\def\switch@tabular{%
 \let\@array@sw\@array@sw@array
 \@ifx{\@array\@array@LaTeX}{%
  \@ifx{\multicolumn\multicolumn@LaTeX}{%
   \@ifx{\@tabular\@tabular@LaTeX}{%
    \@ifx{\@tabarray\@tabarray@LaTeX}{%
     \@ifx{\array\array@LaTeX}{%
      \@ifx{\endarray\endarray@LaTeX}{%
       \@ifx{\endtabular\endtabular@LaTeX}{%
        \@ifx{\@mkpream\@mkpream@LaTeX}{%
         \@ifx{\@addamp\@addamp@LaTeX}{%
          \@ifx{\@arrayacol\@arrayacol@LaTeX}{%
           \@ifx{\@tabacol\@tabacol@LaTeX}{%
            \@ifx{\@arrayclassz\@arrayclassz@LaTeX}{%
             \@ifx{\@tabclassiv\@tabclassiv@LaTeX}{%
              \@ifx{\@arrayclassiv\@arrayclassiv@LaTeX}{%
               \@ifx{\@tabclassz\@tabclassz@LaTeX}{%
                \@ifx{\@classv\@classv@LaTeX}{%
                 \@ifx{\hline\hline@LaTeX}{%
                  \@ifx{\@tabularcr\@tabularcr@LaTeX}{%
                   \@ifx{\@xtabularcr\@xtabularcr@LaTeX}{%
                    \@ifx{\@xargarraycr\@xargarraycr@LaTeX}{%
                     \@ifx{\@yargarraycr\@yargarraycr@LaTeX}{%
                      \true@sw
                     }{%
                      \false@sw
                     }%
                    }{%
                     \false@sw
                    }%
                   }{%
                    \false@sw
                   }%
                  }{%
                   \false@sw
                  }%
                 }{%
                  \false@sw
                 }%
                }{%
                 \false@sw
                }%
               }{%
                \false@sw
               }%
              }{%
               \false@sw
              }%
             }{%
              \false@sw
             }%
            }{%
             \false@sw
            }%
           }{%
            \false@sw
           }%
          }{%
           \false@sw
          }%
         }{%
          \false@sw
         }%
        }{%
         \false@sw
        }%
       }{%
        \false@sw
       }%
      }{%
       \false@sw
      }%
     }{%
      \false@sw
     }%
    }{%
     \false@sw
    }%
   }{%
    \false@sw
   }%
  }{%
   \false@sw
  }%
 }{%
  \false@sw
 }%
 {%
  \class@info{Patching LaTeX tabular.}%
 }{%
  \class@info{Unrecognized LaTeX tabular. Please update this document class! (Proceeding with fingers crossed.)}%
 }%
 \let\@array\@array@ltx
 \let\multicolumn\multicolumn@ltx
 \let\@tabular\@tabular@ltx
 \let\@tabarray\@tabarray@ltx
 \let\array\array@ltx
 \let\endarray\endarray@ltx
 \let\endtabular\endtabular@ltx
 \let\@mkpream\@mkpream@ltx
 \let\@addamp\@addamp@ltx
 \let\@arrayacol\@arrayacol@ltx
 \let\@tabacol\@tabacol@ltx
 \let\@arrayclassz\@arrayclassz@ltx
 \let\@tabclassiv\@tabclassiv@ltx
 \let\@arrayclassiv\@arrayclassiv@ltx
 \let\@tabclassz\@tabclassz@ltx
 \let\@classv\@classv@ltx
 \let\hline\hline@ltx
 \let\@tabularcr\@tabularcr@ltx
 \let\@xtabularcr\@xtabularcr@ltx
 \let\@xargarraycr\@xargarraycr@ltx
 \let\@yargarraycr\@yargarraycr@ltx
}%
%    \end{macrocode}
% \changes{4.1b}{2008/08/12}{(AO, 505) Try to accommodate \classname{colortbl}.}
%    \begin{macrocode}
\def\switch@array{%
 \@ifpackageloaded{colortbl}{\let\switch@array@info\colortbl@message}{\let\switch@array@info\array@message}%
 \let\@array@sw\@array@sw@LaTeX
 \@ifx{\@array\@array@array}{%
  \@ifx{\@tabular\@tabular@array}{%
   \@ifx{\@tabarray\@tabarray@array}{%
    \@ifx{\array\array@array}{%
     \@ifx{\endarray\endarray@array}{%
      \@ifx{\endtabular\endtabular@array}{%
       \@ifx{\@mkpream\@mkpream@array}{%
        \@ifx{\@classx\@classx@array}{%
         \@ifx{\insert@column\insert@column@array}{%
          \@ifx{\@arraycr\@arraycr@array}{%
           \@ifx{\@xarraycr\@xarraycr@array}{%
            \@ifx{\@xargarraycr\@xargarraycr@array}{%
             \@ifx{\@yargarraycr\@yargarraycr@array}{%
              \true@sw
             }{%
              \false@sw
             }%
            }{%
             \false@sw
            }%
           }{%
            \false@sw
           }%
          }{%
           \false@sw
          }%
         }{%
          \false@sw
         }%
        }{%
         \false@sw
        }%
       }{%
        \false@sw
       }%
      }{%
       \false@sw
      }%
     }{%
      \false@sw
     }%
    }{%
     \false@sw
    }%
   }{%
    \false@sw
   }%
  }{%
   \false@sw
  }%
 }{%
  \false@sw
 }{%
  \class@info{Patching array package.}%
 }{%
  \switch@array@info
 }%
 \let\@array    \@array@array@new
 \let\@@array   \@array % Cosi fan tutti
 \let\@tabular  \@tabular@array@new
 \let\@tabarray \@tabarray@array@new
 \let\array     \array@array@new
 \let\endarray  \endarray@array@new
 \let\endtabular\endtabular@array@new
 \let\@mkpream  \@mkpream@array@new
 \let\@classx   \@classx@array@new
 \let\@arrayacol\@arrayacol@ltx
 \let\@tabacol  \@tabacol@ltx
 \let\insert@column\insert@column@array@new
 \expandafter\let\csname endtabular*\endcsname\endtabular % Cosi fan tutti
 \let\@arraycr  \@arraycr@new
 \let\@xarraycr \@xarraycr@new
 \let\@xargarraycr\@xargarraycr@new
 \let\@yargarraycr\@yargarraycr@new
}%
\def\array@message{%
 \class@info{Unrecognized array package. Please update this document class! (Proceeding with fingers crossed.)}%
}%
\def\colortbl@message{%
 \class@info{colortbl package is loaded. (Proceeding with fingers crossed.)}%
}%
%    \end{macrocode}
% \end{macro}
% \end{macro}
%
% \begin{macro}{\@array@sw}
% The Boolean \cmd\@array@sw\ must be different depending on
% whether the \classname{array} package is loaded.
%    \begin{macrocode}
\def\@array@sw@LaTeX{\@ifx{\\\@tabularcr}}%
\def\@array@sw@array{\@ifx{\d@llarbegin\begingroup}}%
%    \end{macrocode}
% \end{macro}
%
% \begin{macro}{\@tabular}
% \changes{4.1b}{2008/08/12}{(AO, 505) try to accommodate \classname{colortbl}.}
% We provide the old versions of \cmd\@tabular\ along with the respective new versions.
% The change here is to avoid committing to LR mode. That will be done later (as late as possible, naturally).
%
% Compatibility note: I had done \cmd\let\ \cmd\col@sep\ \cmd\@undefined\  here, but
% this was not compatible with \classname{colortbl}. I have removed that statement.
%    \begin{macrocode}
\def\@tabular@LaTeX{%
 \leavevmode
 \hbox\bgroup$%
  \let\@acol\@tabacol
  \let\@classz\@tabclassz
  \let\@classiv\@tabclassiv
  \let\\\@tabularcr
  \@tabarray
}%
\def\@tabular@ltx{%
  \let\@acoll\@tabacoll
  \let\@acolr\@tabacolr
  \let\@acol\@tabacol
  \let\@classz\@tabclassz
  \let\@classiv\@tabclassiv
  \let\\\@tabularcr
  \@tabarray
}%
\def\@tabular@array{%
 \leavevmode
 \hbox\bgroup$%
  \col@sep\tabcolsep
  \let\d@llarbegin\begingroup
  \let\d@llarend\endgroup
  \@tabarray
}%
\def\@tabular@array@new{%
  \let\@acoll\@tabacoll
  \let\@acolr\@tabacolr
  \let\@acol\@tabacol
%    \end{macrocode}
% \let\col@sep\@undefined
%    \begin{macrocode}
  \let\d@llarbegin\begingroup
  \let\d@llarend\endgroup
  \@tabarray
}%
%    \end{macrocode}
% \end{macro}
%
% \begin{macro}{\@tabarray}
% Here we provide old and new versions of the \cmd\@tabarray\ procedure.
% The change here is to parametrize the default vertical alignment,
% which is 'c' in standard \LaTeX.
% Under some circumstances, we want to change this to, say, 'v'.
%
% \changes{4.1b}{2008/08/12}{(AO, 505) try to accommodate \classname{colortbl}.}
% FIXME: must decouple \env{array} and \env{tabular}. Done (it seems).
%
% Note on \classname{colortbl}: this package head-patches \cmd\@tabarray
% with its own command \cmd\CT@start, and tails onto \cmd\endarray\ with \cmd\CT@end.
% It fortuitously does the former at \cmd\AtBeginDocument\ time, and, fortuitously, 
% we do not patch \cmd\endarray, which it overwrites.
%    \begin{macrocode}
\def\@tabarray@LaTeX{%
 \m@th\@ifnextchar[\@array{\@array[c]}%
}%
\def\@tabarray@ltx{%
 \m@th\@ifnextchar[\@array{\expandafter\@array\expandafter[\array@default]}%
}%
\def\@tabarray@array{%
 \@ifnextchar[{\@@array}{\@@array[c]}%
}%
\def\@tabarray@array@new{%
 \@ifnextchar[{\@@array}{\expandafter\@@array\expandafter[\array@default]}%
}%
%    \end{macrocode}
% \end{macro}
%
% \begin{macro}{\@tabularcr}
% \begin{macro}{\@tbpen}
% \begin{macro}{\@tabularcr}
% \begin{macro}{\@xtabularcr}
% \begin{macro}{\@xargarraycr}
% \begin{macro}{\@yargarraycr}
% \begin{macro}{\@arraycr}
% \begin{macro}{\@xarraycr}
% We provide for the \cmd\\ command within \env{tabular} to provide control over page breaking, just the same as
% that of \env{eqnarray}.
%
% The count register \cmd\intertabularlinepenalty\ is similar to \cmd\interdisplaylinepenalty: it is the penalty
% associated with each row of a tabular. When it is set to \cmd\@M, the tabular will cleave together.
%
% The count register \cmd\@tbpen\ is similar to \cmd\@eqpen: it memorizes the penalty to use after the current tabular row.
% If the \cmd\\ command is in its star form, then \cmd\@eqpen\ is set to \cmd\@M.
%
% We append code to \cmd\samepage\ so that a tabular within its scope will cleave together.
%
% We keep the standard definition of \cmd\@tabularcr\ in \cmd\@tabularcr@LaTeX\ for reference,
% and provide a new definition that works like \cmd\@eqncr: it sets \cmd\@tbpen\ to \cmd\@M\ if the star was given.
%
% We also provide new versions of \cmd\@xtabularcr, \cmd\@xargarraycr, and \cmd\@yargarraycr, all of which invoke \cmd\@tbpen.
%
% The \cmd\switch@tabular\ procedure switches in the new definitions.
%    \begin{macrocode}
\newcount\intertabularlinepenalty
\intertabularlinepenalty=100
\newcount\@tbpen
\appdef\samepage{\intertabularlinepenalty\@M}%
\def\@tabularcr@LaTeX{{\ifnum 0=`}\fi \@ifstar \@xtabularcr \@xtabularcr}%
\def\@tabularcr@ltx{{\ifnum 0=`}\fi \@ifstar {\global \@tbpen \@M \@xtabularcr }{\global \@tbpen \intertabularlinepenalty \@xtabularcr }}%
\def\@xtabularcr@LaTeX{\@ifnextchar [\@argtabularcr {\ifnum 0=`{\fi }\cr }}%
\def\@xtabularcr@ltx{\@ifnextchar [\@argtabularcr {\ifnum 0=`{\fi }\cr \noalign {\penalty \@tbpen }}}%
\def\@xargarraycr@LaTeX#1{\@tempdima #1\advance \@tempdima \dp \@arstrutbox \vrule \@height \z@ \@depth \@tempdima \@width \z@ \cr}%
\def\@xargarraycr@ltx#1{\@tempdima #1\advance \@tempdima \dp \@arstrutbox \vrule \@height \z@ \@depth \@tempdima \@width \z@ \cr \noalign {\penalty \@tbpen }}%
\def\@yargarraycr@LaTeX#1{\cr \noalign {\vskip #1}}%
\def\@yargarraycr@ltx#1{\cr \noalign {\penalty \@tbpen \vskip #1}}%
%    \end{macrocode}
%
% If the \classname{array} package has been loaded, we must alter the meanings of 
% \cmd\@arraycr, \cmd\@xarraycr, \cmd\@xargarraycr, and \cmd\@yargarraycr.
% In this case, it is \cmd\switch@array\ that switches in the new definitions.
%    \begin{macrocode}
\def\@arraycr@array{%
 \relax
 \iffalse{\fi\ifnum 0=`}\fi
 \@ifstar \@xarraycr \@xarraycr
}%
\def\@arraycr@new{%
 \relax
 \iffalse{\fi\ifnum 0=`}\fi
 \@ifstar {\global \@tbpen \@M \@xarraycr }{\global \@tbpen \intertabularlinepenalty \@xarraycr }%
}%
\def\@xarraycr@array{%
 \@ifnextchar [%]
 \@argarraycr {\ifnum 0=`{}\fi\cr}%
}%
\def\@xarraycr@new{%
 \@ifnextchar [%]
 \@argarraycr {\ifnum 0=`{}\fi\cr \noalign {\penalty \@tbpen }}%
}%
\def\@xargarraycr@array#1{%
 \unskip
 \@tempdima #1\advance\@tempdima \dp\@arstrutbox
 \vrule \@depth\@tempdima \@width\z@
 \cr
}%
\def\@xargarraycr@new#1{%
 \unskip
 \@tempdima #1\advance\@tempdima \dp\@arstrutbox
 \vrule \@depth\@tempdima \@width\z@
 \cr
 \noalign {\penalty \@tbpen }%
}%
\def\@yargarraycr@array#1{%
 \cr
 \noalign{\vskip #1}%
}%
\def\@yargarraycr@new#1{%
 \cr
 \noalign{\penalty \@tbpen \vskip #1}%
}%
%    \end{macrocode}
% \end{macro}
% \end{macro}
% \end{macro}
% \end{macro}
% \end{macro}
% \end{macro}
% \end{macro}
% \end{macro}
%
% \begin{macro}{\array}
% We provide old and new versions of the \cmd\array\ procedure for both \LaTeX\ and the \classname{array} package.
% The change here is to accomodate the new procedures that will be called for the array boundaries, even
% though at present they are not special.
% A thought: here is where matrices can be readily accomodated.
%    \begin{macrocode}
\def\array@LaTeX{%
 \let\@acol\@arrayacol
 \let\@classz\@arrayclassz
 \let\@classiv\@arrayclassiv
 \let\\\@arraycr
 \let\@halignto\@empty
 \@tabarray
}%
\def\array@ltx{%
 \@ifmmode{}{\@badmath$}%
 \let\@acoll\@arrayacol
 \let\@acolr\@arrayacol
 \let\@acol\@arrayacol
 \let\@classz\@arrayclassz
 \let\@classiv\@arrayclassiv
 \let\\\@arraycr
 \let\@halignto\@empty
 \@tabarray
}%
\def\array@array{%
 \col@sep\arraycolsep
 \def\d@llarbegin{$}\let\d@llarend\d@llarbegin\gdef\@halignto{}%
 \@tabarray
}
\def\array@array@new{%
 \@ifmmode{}{\@badmath$}%
 \let\@acoll\@arrayacol
 \let\@acolr\@arrayacol
 \let\@acol\@arrayacol
%    \end{macrocode}
% \changes{4.1b}{2008/08/12}{(AO, 505) try to accommodate \classname{colortbl}.}
% Removed: \verb+\let\col@sep\@undefined+
%    \begin{macrocode}
 \def\d@llarbegin{$}%
 \let\d@llarend\d@llarbegin
 \gdef\@halignto{}%
 \@tabarray
}%
%    \end{macrocode}
% \end{macro}
%
% \begin{macro}{\@array}
% Here we provide old and new versions of \cmd\@array.
% The change here is to provide a convenient, flexible, and extensible
% mechanism for new vertical alignment options.
%
% Instead of testing the optional argument with \cmd\if, we
% use a dispatcher based on \cmd\csname.
%
% We also refrain from using \cmd\ialign, which would set
% the \cmd\tabskip\ to the wrong value.
%
% Finally, the procedure to set the \cmd\@arstrutbox\
% is broken out so that it can be patched.
%    \begin{macrocode}
\def\@array@LaTeX[#1]#2{%
  \if #1t\vtop \else \if#1b\vbox \else \vcenter \fi\fi
  \bgroup
  \setbox\@arstrutbox\hbox{%
    \vrule \@height\arraystretch\ht\strutbox
           \@depth\arraystretch \dp\strutbox
           \@width\z@}%
  \@mkpream{#2}%
  \edef\@preamble{%
    \ialign \noexpand\@halignto
      \bgroup \@arstrut \@preamble \tabskip\z@skip \cr}%
  \let\@startpbox\@@startpbox \let\@endpbox\@@endpbox
  \let\tabularnewline\\%
    \let\par\@empty
    \let\@sharp##%
    \set@typeset@protect
    \lineskip\z@skip\baselineskip\z@skip
    \ifhmode \@preamerr\z@ \@@par\fi
    \@preamble
}%
\def\@array@ltx[#1]#2{%
 \@nameuse{@array@align@#1}%
  \set@arstrutbox
  \@mkpream{#2}%
  \prepdef\@preamble{%
    \tabskip\tabmid@skip
    \@arstrut
  }%
  \appdef\@preamble{%
    \tabskip\tabright@skip
    \cr
    \array@row@pre
  }%
% \let\@startpbox\@@startpbox
% \let\@endpbox\@@endpbox
  \let\tabularnewline\\%
  \let\par\@empty
  \let\@sharp##%
  \set@typeset@protect
  \lineskip\z@skip\baselineskip\z@skip
  \tabskip\tableft@skip\relax
  \ifhmode \@preamerr\z@ \@@par\fi
  \everycr{}%
  \expandafter\halign\expandafter\@halignto\expandafter\bgroup\@preamble
}%
%
\def\set@arstrutbox{%
  \setbox\@arstrutbox\hbox{%
    \vrule \@height\arraystretch\ht\strutbox
           \@depth\arraystretch \dp\strutbox
           \@width\z@
  }%
}%
%    \end{macrocode}
% \end{macro}
%
% \begin{macro}{\@array@array}
% 
%    \begin{macrocode}
\def\@array@array[#1]#2{%
  \@tempdima \ht \strutbox
  \advance \@tempdima by\extrarowheight
  \setbox \@arstrutbox \hbox{\vrule
             \@height \arraystretch \@tempdima
             \@depth \arraystretch \dp \strutbox
             \@width \z@}%
  \begingroup
  \@mkpream{#2}%
  \xdef\@preamble{\noexpand \ialign \@halignto
                  \bgroup \@arstrut \@preamble
                          \tabskip \z@ \cr}%
  \endgroup
  \@arrayleft
  \if #1t\vtop \else \if#1b\vbox \else \vcenter \fi \fi
  \bgroup
  \let \@sharp ##\let \protect \relax
  \lineskip \z@
  \baselineskip \z@
  \m@th
  \let\\\@arraycr \let\tabularnewline\\\let\par\@empty \@preamble
}%
\def\@array@array@new[#1]#2{%
  \@tempdima\ht\strutbox
  \advance\@tempdima by\extrarowheight
  \setbox\@arstrutbox\hbox{%
   \vrule \@height\arraystretch\@tempdima
          \@depth \arraystretch\dp\strutbox
          \@width \z@
  }%
  \begingroup
   \@mkpream{#2}%
   \xdef\@preamble{\@preamble}%
  \endgroup
  \prepdef\@preamble{%
   \tabskip\tabmid@skip
    \@arstrut
  }%
  \appdef\@preamble{%
   \tabskip\tabright@skip
   \cr
   \array@row@pre
  }%
  \@arrayleft
  \@nameuse{@array@align@#1}%
  \m@th
  \let\\\@arraycr
  \let\tabularnewline\\%
  \let\par\@empty
  \let\@sharp##%
  \set@typeset@protect
  \lineskip\z@\baselineskip\z@
  \tabskip\tableft@skip
  \everycr{}%
  \expandafter\halign\expandafter\@halignto\expandafter\bgroup\@preamble
}%
%    \end{macrocode}
% \end{macro}
%
% \begin{macro}{\endarray}
% Here we provide old and new versions of \cmd\endarray.
% The change here is to use a single procedure to close
% out any array-like structure, namely \cmd\endarray@ltx.
% It merely closes out the \cmd\halign.
%    \begin{macrocode}
\def\endarray@LaTeX{%
 \crcr\egroup\egroup
}%
\def\endarray@ltx{%
 \crcr\array@row@pst\egroup\egroup
}%
\def\endarray@array{%
 \crcr \egroup \egroup \@arrayright \gdef\@preamble{}%
}%
\def\endarray@array@new{%
 \crcr\array@row@pst\egroup\egroup % Same as \endarray@ltx
 \@arrayright
 \global\let\@preamble\@empty
}%
%    \end{macrocode}
% \end{macro}
%
% \begin{macro}{\endtabular}
% 
%    \begin{macrocode}
\def\endtabular@LaTeX{%
 \crcr\egroup\egroup $\egroup
}%
\def\endtabular@ltx{%
 \endarray
}%
\def\endtabular@array{%
 \endarray $\egroup
}%
\def\endtabular@array@new{%
 \endarray
}%
%    \end{macrocode}
% \end{macro}
%
% \begin{macro}{endtabular*}
% Here we provide a proper definition for the star-form of \enve{endtabular}.
% It is one of the enduring curiosities that the \LaTeX\ kernel continues to use
% dangerously and inappropriately ``optimized'' definitions for such commands.
%    \begin{macrocode}
\@namedef{endtabular*}{\endtabular}%
%    \end{macrocode}
% \end{macro}
%
% \begin{macro}{\multicolumn}
% 
%    \begin{macrocode}
\long\def\multicolumn@LaTeX#1#2#3{%
 \multispan{#1}\begingroup
  \@mkpream{#2}%
  \def\@sharp{#3}\set@typeset@protect
  \let\@startpbox\@@startpbox\let\@endpbox\@@endpbox
  \@arstrut \@preamble\hbox{}\endgroup\ignorespaces
}%
\long\def\multicolumn@ltx#1#2#3{%
 \multispan{#1}%
 \begingroup
  \@mkpream{#2}%
  \def\@sharp{#3}%
  \set@typeset@protect
 %\let\@startpbox\@@startpbox\let\@endpbox\@@endpbox
  \@arstrut
  \@preamble
  \hbox{}%
 \endgroup
 \ignorespaces
}%
%    \end{macrocode}
% \end{macro}
%
% \begin{macro}{\@array@align@}
% \begin{macro}{\array@default}
% Here are the various procedures for the vertical alignment options.
% The change from standard \LaTeX\ is that we do not go into math mode
% in every case: only when required by \cmd\vcenter. 
% Also, we use \cmd\aftergroup\ to close out the boxes and modes we have started.
% It requires only that each procedure issue exactly one unmatched \cmd\bgroup.
%
% We establish here the default vertical alignment.
%    \begin{macrocode}
\def\@array@align@t{\leavevmode\vtop\bgroup}%
\def\@array@align@b{\leavevmode\vbox\bgroup}%
\def\@array@align@c{\leavevmode\@ifmmode{\vcenter\bgroup}{$\vcenter\bgroup\aftergroup$\aftergroup\relax}}%
\def\@array@align@v{%
 \@ifmmode{%
  \@badmath
  \vcenter\bgroup
 }{%
  \@ifinner{%
   $\vcenter\bgroup\aftergroup$
  }{%
   \@@par\bgroup
  }%
 }%
}%
\def\array@default{c}%
%    \end{macrocode}
% \end{macro}
% \end{macro}
%
% \begin{macro}{\array@row@pre}
% \begin{macro}{\array@row@pst}
% \begin{macro}{\array@row@rst}
% The procedure \cmd\array@row@rst\ reestablishes a default context for
% an alignment, so that they can be nested.
% Any environment or procedure that alters the way alignments are formatted
% must patch this procedure to restore from that alteration.
% To start things off, we equate \cmd\@array@align@v\ to \cmd\@array@align@c,
% because it does not make sense to do the former in any context other 
% than the MVL or in a list that will be unboxed onto the MVL.
%    \begin{macrocode}
\def\array@row@rst{%
 \let\@array@align@v\@array@align@c
}%
\def\array@row@pre{}%
\def\array@row@pst{}%
%    \end{macrocode}
% \end{macro}
% \end{macro}
% \end{macro}
%
% \begin{macro}{\toprule}
% \begin{macro}{\colrule}
% \begin{macro}{\botrule}
% Default definitions for \cmd\toprule, \cmd\colrule, \cmd\botrule
%    \begin{macrocode}
\newcommand\toprule{\tab@rule{\column@font}{\column@fil}{\frstrut}}%
\newcommand\colrule{\unskip\lrstrut\\\tab@rule{\body@font}{}{\frstrut}}%
\newcommand\botrule{\unskip\lrstrut\\\noalign{\hline@rule}{}}%
%    \end{macrocode}
% \end{macro}
% \end{macro}
% \end{macro}
%
% \begin{macro}{\hline}
%    \begin{macrocode}
\def\hline@LaTeX{%
 \noalign{\ifnum0=`}\fi\hrule \@height \arrayrulewidth \futurelet
   \reserved@a\@xhline
}%
\def\hline@ltx{%
 \noalign{%
  \ifnum0=`}\fi
  \hline@rule
  \futurelet\reserved@a\@xhline
 % \noalign ended in \@xhline
}%
\def\@xhline@unneeded{%
 \say\reserved@a
 \ifx\reserved@a\hline
  \vskip\doublerulesep
  \vskip-\arrayrulewidth
 \fi
 \ifnum0=`{\fi}%
}%
\def\tab@rule#1#2#3{%
 \crcr
 \noalign{%
  \hline@rule
  \gdef\@arstrut@hook{%
   \global\let\@arstrut@hook\@empty
   #3%
  }%
  \gdef\cell@font{#1}%
  \gdef\cell@fil{#2}%
 }%
}%
\def\column@font{}%
\def\column@fil{}%
\def\body@font{}%
\def\cell@font{}%
\def\frstrut{}%
\def\lrstrut{}%
%    \end{macrocode}
% \end{macro}
%
% \begin{macro}{\@arstrut@hline}
% \begin{macro}{\@arstrut@org}
% \begin{macro}{\@arstrut@hook}
% \begin{macro}{\@arstrutbox@hline}
% \begin{macro}{\set@arstrutbox}
% \begin{macro}{\hline@rule}
% The procedure \cmd\@arstrut@hline\ is substantially the same as
% \cmd\@arstrut, except the strut copied in is \cmd\@arstrutbox@hline
% instead of \cmd\@arstrutbox.
%
% The procedure \cmd\@arstrut@hook\ is redefined in \cmd\tab@rule!
%
% The register \cmd\@arstrutbox@hline.
%
% We append to \cmd\set@arstrutbox\ the code necessary to set a strut following an \cmd\hline.
%
% The procedure \cmd\hline@rule\ lays down a rule, and changes the meaning of \cmd\@arstrut\
% so that the next line will be correctly strutted.
%
% The \cmd\@arstrut@hline@clnc\ is a klootch, a magic number.
%    \begin{macrocode}
\def\@arstrut@hline{%
 \relax
 \@ifmmode{\copy}{\unhcopy}\@arstrutbox@hline
 \@arstrut@hook
}%
%
\let\@arstrut@org\@arstrut
\def\@arstrut@hook{%
 \global\let\@arstrut\@arstrut@org
}%
%
\newbox\@arstrutbox@hline
\appdef\set@arstrutbox{%
  \setbox\@arstrutbox@hline\hbox{%
    \setbox\z@\hbox{$0^{0}_{}$}%
    \dimen@\ht\z@\advance\dimen@\@arstrut@hline@clnc
    \@ifdim{\dimen@<\arraystretch\ht\strutbox}{\dimen@=\arraystretch\ht\strutbox}{}%
    \vrule \@height\dimen@
           \@depth\arraystretch \dp\strutbox
           \@width\z@
  }%
}%
%
\def\hline@rule{%
 \hrule \@height \arrayrulewidth
 \global\let\@arstrut\@arstrut@hline
}%
\def\@arstrut@hline@clnc{2\p@}% % Klootch: magic number
%    \end{macrocode}
% \end{macro}
% \end{macro}
% \end{macro}
% \end{macro}
% \end{macro}
% \end{macro}
%
% \begin{macro}{\tableft@skip}
%    \begin{macrocode}
\def\tableft@skip{\z@skip}%
\def\tabmid@skip{\z@skip}%\@flushglue
\def\tabright@skip{\z@skip}%
\def\tableftsep{\tabcolsep}%
\def\tabmidsep{\tabcolsep}%
\def\tabrightsep{\tabcolsep}%
\def\cell@fil{}%
\def\pbox@hook{}%
%    \end{macrocode}
% \end{macro}
%
% \begin{macro}{\@arstrut}
%    \begin{macrocode}
\appdef\@arstrut{\@arstrut@hook}%
\let\@arstrut@hook\@empty
\def\@addtopreamble{\appdef\@preamble}%
%    \end{macrocode}
% \end{macro}
%
% \begin{macro}{\@mkpream}
%    \begin{macrocode}
\def\@mkpream@LaTeX#1{%
  \@firstamptrue\@lastchclass6
  \let\@preamble\@empty
  \let\protect\@unexpandable@protect
  \let\@sharp\relax
  \let\@startpbox\relax\let\@endpbox\relax
  \@expast{#1}%
  \expandafter\@tfor \expandafter
    \@nextchar \expandafter:\expandafter=\reserved@a\do
       {\@testpach\@nextchar
    \ifcase \@chclass \@classz \or \@classi \or \@classii \or \@classiii
      \or \@classiv \or\@classv \fi\@lastchclass\@chclass}%
  \ifcase \@lastchclass \@acol
      \or \or \@preamerr \@ne\or \@preamerr \tw@\or \or \@acol \fi
}%
\def\@mkpream@ltx#1{%
 \@firstamptrue
 \@lastchclass6
 \let\@preamble\@empty
 \let\protect\@unexpandable@protect
 \let\@sharp\relax
%\let\@startpbox\relax\let\@endpbox\relax
 \@expast{#1}%
 \expandafter\@tfor\expandafter\@nextchar\expandafter:\expandafter=\reserved@a
 \do{%
  \expandafter\@testpach\expandafter{\@nextchar}%
  \ifcase\@chclass
   \@classz
  \or
   \@classi
  \or
   \@classii
  \or
   \@classiii
  \or
   \@classiv
  \or
   \@classv
  \fi
  \@lastchclass\@chclass
 }%
 \ifcase\@lastchclass
  \@acolr % right-hand column
 \or
 \or
  \@preamerr\@ne
 \or
  \@preamerr\tw@
 \or
 \or
  \@acolr % right-hand column
 \fi
}%
%    \end{macrocode}
% \end{macro}
%
% \begin{macro}{\insert@column}
%    \begin{macrocode}
\def\insert@column@array{%
   \the@toks \the \@tempcnta
   \ignorespaces \@sharp \unskip
   \the@toks \the \count@ \relax
}%
\def\insert@column@array@new{%
 \the@toks\the\@tempcnta
 \array@row@rst\cell@font
 \ignorespaces\@sharp\unskip
 \the@toks\the\count@
 \relax
}%
%    \end{macrocode}
% \end{macro}
%
% \begin{macro}{\@mkpream@relax}
% The procedure \cmd\@mkpream@relax\ participates in a strange and wonderful
% method of binding the alignment procedure---but only certain parts thereof.
%
% Here is how it works: in \LaTeX, the \classname{array} package, and in the
% \classname{longtable} package alike, there is a need to create an alignment
% preamble (using \cmd\@mkpream) for use by the upcoming \cmd\halign.
% Then, in both \classname{array} and \classname{longtable}, \TeX's \cmd\edef\
% is used to `compile in place' that alignment preamble. 
%
% In the case of \classname{array}, 
% the operation is done in order to pre-expand the use of \texttt{*}; 
% in \classname{longtable}, it is to set the widths of the columns.
%
% Now, during this \cmd\edef, certain control sequence names must \emph{not}
% be expanded, and those are robustified by \cmd\@mkpream@relax.
%
%    \begin{macrocode}
\def\@mkpream@relax{%
 \let\tableftsep   \relax
 \let\tabmidsep    \relax
 \let\tabrightsep  \relax
 \let\array@row@rst\relax
 \let\cell@font    \relax
 \let\@startpbox   \relax
}%
%    \end{macrocode}
% \end{macro}
%
% \begin{macro}{\@mkpream}
% We insert \cmd\@mkpream@relax\ at the head of the procedure.
% The robustifying of \cmd\@startpbox\ and \cmd\@endpbox\ is taken over by this mechanism.
% We also invoke \cmd\@acolr\ instead of \cmd\@acol\ when a right-hand column is at hand.
%
% \changes{4.1b}{2008/08/12}{(AO, 505) try to accommodate \classname{colortbl}.}
% Note on \classname{colortbl}: this package head-patches \cmd\@mkpream\ to robustify
% a number of its commands during the construction of the alignment preamble.
% The best we can do is to supplement the \cmd\@mkpream@relax\ procedure to perform this action.
%    \begin{macrocode}
\def\@mkpream@array#1{%
   \gdef\@preamble{}\@lastchclass 4 \@firstamptrue
   \let\@sharp\relax \let\@startpbox\relax \let\@endpbox\relax
   \@temptokena{#1}\@tempswatrue
   \@whilesw\if@tempswa\fi{\@tempswafalse\the\NC@list}%
   \count@\m@ne
   \let\the@toks\relax
   \prepnext@tok
   \expandafter \@tfor \expandafter \@nextchar
    \expandafter :\expandafter =\the\@temptokena \do
   {\@testpach
   \ifcase \@chclass \@classz \or \@classi \or \@classii
     \or \save@decl \or \or \@classv \or \@classvi
     \or \@classvii \or \@classviii
     \or \@classx
     \or \@classx \fi
   \@lastchclass\@chclass}%
   \ifcase\@lastchclass
   \@acol \or
   \or
   \@acol \or
   \@preamerr \thr@@ \or
   \@preamerr \tw@ \@addtopreamble\@sharp \or
   \or
   \else  \@preamerr \@ne \fi
   \def\the@toks{\the\toks}%
}%
\def\@mkpream@array@new#1{%
 \gdef\@preamble{}%
 \@lastchclass\f@ur
 \@firstamptrue
 \let\@sharp\relax
 \@mkpream@relax
%\let\@startpbox\relax\let\@endpbox\relax
 \@temptokena{#1}\@tempswatrue
 \@whilesw\if@tempswa\fi{\@tempswafalse\the\NC@list}%
 \count@\m@ne
 \let\the@toks\relax
 \prepnext@tok
 \expandafter\@tfor\expandafter\@nextchar\expandafter:\expandafter=\the\@temptokena
 \do{%
  \@testpach
  \ifcase\@chclass
   \@classz
  \or
   \@classi
  \or
   \@classii
  \or
   \save@decl
  \or
  \or
   \@classv
  \or
   \@classvi
  \or
   \@classvii
  \or
   \@classviii
  \or
   \@classx
  \or
   \@classx
  \fi
  \@lastchclass\@chclass
 }%
 \ifcase\@lastchclass
  \@acolr % right-hand column
 \or
 \or
  \@acolr % right-hand column
 \or
  \@preamerr\thr@@
 \or
  \@preamerr\tw@\@addtopreamble\@sharp
 \or
 \or
 \else
  \@preamerr\@ne
 \fi
 \def\the@toks{\the\toks}%
}%
%    \end{macrocode}
% \end{macro}
%
% \begin{macro}{\@mkpream@relax}
% \changes{4.1b}{2008/08/12}{(AO, 505) try to accommodate \classname{colortbl}.}
% David P. Carlisle's \classname{colortbl} package headpatches \cmd\@mkpream\ in place
% during package loading, so it does not know whom it is working on.
% Let us try to accomodate this package by doing what it would have liked to have done.
%
% Note: it would be far better to break out this mechanism in the \classname{array} package.
%    \begin{macrocode}
\appdef\@mkpream@relax{%
 \let\CT@setup       \relax
 \let\CT@color       \relax
 \let\CT@do@color    \relax
 \let\color          \relax
 \let\CT@column@color\relax
 \let\CT@row@color   \relax
 \let\CT@cell@color  \relax
}%
%    \end{macrocode}
% \end{macro}
%
% \begin{macro}{\@addamp}
%    \begin{macrocode}
\def\@addamp@LaTeX{%
  \if@firstamp\@firstampfalse\else\edef\@preamble{\@preamble &}\fi
}%
\def\@addamp@ltx{%
 \if@firstamp\@firstampfalse\else\@addtopreamble{&}\fi
}%
%    \end{macrocode}
% \end{macro}
%
% \begin{macro}{\@arrayacol}
%    \begin{macrocode}
\def\@arrayacol@LaTeX{%
 \edef\@preamble{\@preamble \hskip \arraycolsep}%
}%
\def\@arrayacol@ltx{%
 \@addtopreamble{\hskip\arraycolsep}%
}%
%    \end{macrocode}
% \end{macro}
%
% \begin{macro}{\@tabacol}
%    \begin{macrocode}
\def\@tabacoll{%
 \@addtopreamble{\hskip\tableftsep\relax}%
}%
\def\@tabacol@LaTeX{%
 \edef\@preamble{\@preamble \hskip \tabcolsep}%
}%
\def\@tabacol@ltx{%
 \@addtopreamble{\hskip\tabmidsep\relax}%
}%
\def\@tabacolr{%
 \@addtopreamble{\hskip\tabrightsep\relax}%
}%
%    \end{macrocode}
% \end{macro}
%
% \begin{macro}{\@arrayclassz}
%    \begin{macrocode}
\def\@arrayclassz@LaTeX{%
 \ifcase \@lastchclass \@acolampacol \or \@ampacol \or
   \or \or \@addamp \or
   \@acolampacol \or \@firstampfalse \@acol \fi
 \edef\@preamble{\@preamble
  \ifcase \@chnum
     \hfil$\relax\@sharp$\hfil \or $\relax\@sharp$\hfil
    \or \hfil$\relax\@sharp$\fi}%
}%
\def\@arrayclassz@ltx{%
 \ifcase\@lastchclass
  \@acolampacol
 \or
  \@ampacol
 \or
 \or
 \or
  \@addamp
 \or
  \@acolampacol
 \or
  \@firstampfalse\@acoll
 \fi
 \ifcase\@chnum
  \@addtopreamble{%
   \hfil\array@row@rst$\relax\@sharp$\hfil
  }%
 \or
  \@addtopreamble{%
   \array@row@rst$\relax\@sharp$\hfil
  }%
 \or
  \@addtopreamble{%
   \hfil\array@row@rst$\relax\@sharp$%
  }%
 \fi
}%
%    \end{macrocode}
% \end{macro}
%
% \begin{macro}{\@tabclassz}
%    \begin{macrocode}
\def\@tabclassz@LaTeX{%
  \ifcase\@lastchclass
    \@acolampacol
  \or
    \@ampacol
  \or
  \or
  \or
    \@addamp
  \or
    \@acolampacol
  \or
    \@firstampfalse\@acol
  \fi
  \edef\@preamble{%
    \@preamble{%
      \ifcase\@chnum
        \hfil\ignorespaces\@sharp\unskip\hfil
      \or
        \hskip1sp\ignorespaces\@sharp\unskip\hfil
      \or
        \hfil\hskip1sp\ignorespaces\@sharp\unskip
      \fi}}%
}%
\def\@tabclassz@ltx{%
 \ifcase\@lastchclass
  \@acolampacol
 \or
  \@ampacol
 \or
 \or
 \or
  \@addamp
 \or
  \@acolampacol
 \or
  \@firstampfalse\@acoll
 \fi
 \ifcase\@chnum
  \@addtopreamble{%
   {\hfil\array@row@rst\cell@font\ignorespaces\@sharp\unskip\hfil}%
  }%
 \or
  \@addtopreamble{%
   {\cell@fil\hskip1sp\array@row@rst\cell@font\ignorespaces\@sharp\unskip\hfil}%
  }%
 \or
  \@addtopreamble{%
   {\hfil\hskip1sp\array@row@rst\cell@font\ignorespaces\@sharp\unskip\cell@fil}%
  }%
 \fi
}%
%    \end{macrocode}
% \end{macro}
%
% \begin{macro}{\@tabclassiv}
%    \begin{macrocode}
\def\@tabclassiv@LaTeX{%
 \@addtopreamble\@nextchar
}%
\def\@tabclassiv@ltx{%
 \expandafter\@addtopreamble\expandafter{\@nextchar}%
}%
%    \end{macrocode}
% \end{macro}
%
% \begin{macro}{\@arrayclassiv}
%    \begin{macrocode}
\def\@arrayclassiv@LaTeX{%
 \@addtopreamble{$\@nextchar$}%
}%
\def\@arrayclassiv@ltx{%
 \expandafter\@addtopreamble\expandafter{\expandafter$\@nextchar$}%
}%
%    \end{macrocode}
% \end{macro}
%
% \begin{macro}{\@classv}
%    \begin{macrocode}
\def\@classv@LaTeX{%
 \@addtopreamble{\@startpbox{\@nextchar}\ignorespaces
 \@sharp\@endpbox}%
}%
\def\@classv@ltx{%
 \expandafter\@addtopreamble
 \expandafter{%
 \expandafter \@startpbox
 \expandafter {\@nextchar}%
 \pbox@hook\array@row@rst\cell@font\ignorespaces\@sharp\@endpbox
 }%
}%
%    \end{macrocode}
% \end{macro}
%
% \begin{macro}{\@classx}
%    \begin{macrocode}
\def\@classx@array{%
  \ifcase \@lastchclass
  \@acolampacol \or
  \@addamp \@acol \or
  \@acolampacol \or
  \or
  \@acol \@firstampfalse \or
  \@addamp
  \fi
}%
\def\@classx@array@new{%
 \ifcase \@lastchclass
  \@acolampacol
 \or
  \@addamp \@acol
 \or
  \@acolampacol
 \or
 \or
  \@firstampfalse\@acoll
 \or
  \@addamp
 \fi
}%
%    \end{macrocode}
% \end{macro}
%
%
% \subsection{Repair other broken parts of \LaTeX}
%
% \begin{macro}{\@xbitor}
% Expansion part has extraneous space token. Removed.
%    \begin{macrocode}
\def\@xbitor@LaTeX #1{\@tempcntb \count#1
   \ifnum \@tempcnta =\z@
   \else
     \divide\@tempcntb\@tempcnta
     \ifodd\@tempcntb \@testtrue\fi
   \fi}%
\def\@xbitor@ltx#1{%
 \@tempcntb\count#1\relax
 \@ifnum{\@tempcnta=\z@}{}{%
  \divide\@tempcntb\@tempcnta
  \@ifodd\@tempcntb{\@testtrue}{}%
 }%
}%
\@ifx{\@xbitor\@xbitor@LaTeX}{%
  \class@info{Repairing broken LaTeX \string\@xbitor}%
}{%
  \class@info{Unrecognized LaTeX \string\@xbitor. Please update this document class! (Proceeding with fingers crossed.)}%
}%
\let\@xbitor\@xbitor@ltx
%    \end{macrocode}
% \end{macro}
%
%
% \subsection{Syntax}
% \begin{macro}{\@gobble@opt@one}
% The \cmd\@gobble@opt@one\ command eats up an optional argument
% and one required argument.
%    \begin{macrocode}
\newcommand*\@gobble@opt@one[2][]{}%
%    \end{macrocode}
% \end{macro}
%
% \subsection{Auto-indented Contents}
% Facility to automatically determine the proper indentation of
% the TOC entries.
%
% Note on \classname{hyperref} compatibility:
% We must respect that
% \cmd\contentsline\ now has a fourth argument.
% So, instead of trying to override the meaning of \cmd\contentsline,
% we use the aux file to remember max values from one run to the next.
%
% In this respect, this package retains compatibility with 
% \classname{hyperref}.
%
% \begin{macro}{\@starttoc}
% Install hooks at beginning and end of the TOC processing.
%    \begin{macrocode}
\def\@starttoc#1{%
  \begingroup
    \toc@pre
    \makeatletter
    \@input{\jobname.#1}%
    \if@filesw
      \expandafter\newwrite\csname tf@#1\endcsname
      \immediate\openout \csname tf@#1\endcsname \jobname.#1\relax
    \fi
    \@nobreakfalse
    \toc@post
  \endgroup
}%
\def\toc@pre{}%
\def\toc@post{}%
%    \end{macrocode}
% \end{macro}
%
% \begin{macro}{\toc@@font}
% Interface for setting the formatting characteristics of this part
%  of the TOC.
%
% Note: \cmd\toc@@font\ is the common font for all auto-sizing toc commands,
% although this, too, could become a dispatcher.
% \changes{4.1a}{2008/01/19}{(AO, 461) Change the csname from \cs{@dotsep} to \cs{ltxu@dotsep}. The former is understood in mu. (What we wanted was a dimension.)}%
%    \begin{macrocode}
\def\toc@@font{}%
\def\ltxu@dotsep{\z@}%
%    \end{macrocode}
% \end{macro}
%
% \begin{macro}{\l@section}
% Interface for determining which TOC elements are automatically indented.
%
% All of the \cmd\l@\dots\ commands simply go through the 
% utility procedure \cmd\l@@sections. The calling convention is
% to pass the name of self and the name of parent.
% If you want to exclude any of these from the indentation
% scheme, simply leave the \cmd\l@\dots\ command undefined.
%
% Note that the parent of ``section'' is nil, so we have to define a stub.
% \begin{verbatim}\def\l@section{\l@@sections{}{section}}% Implicit #3#4\end{verbatim}
% \begin{verbatim}\def\tocleft@{\z@}%\end{verbatim}
% \begin{verbatim}\def\l@subsection{\l@@sections{section}{subsection}}% Implicit #3#4\end{verbatim}
% \begin{verbatim}\def\l@subsubsection{\l@@sections{subsection}{subsubsection}}% Implicit #3#4\end{verbatim}
% \begin{verbatim}\def\l@paragraph{\l@@sections{subsubsection}{paragraph}}% Implicit #3#4\end{verbatim}
% \begin{verbatim}\def\l@subparagraph#1#2{\l@@sections{paragraph}{subparagraph}}% Implicit #3#4\end{verbatim}
% \end{macro}
%
% Glom some \cmd\dimen\ registers.
%    \begin{macrocode}
\let\tocdim@section       \leftmargini
\let\tocdim@subsection    \leftmarginii
\let\tocdim@subsubsection \leftmarginiii
\let\tocdim@paragraph     \leftmarginiv
\let\tocdim@appendix      \leftmarginv
\let\tocdim@pagenum       \leftmarginvi
%    \end{macrocode}
%
% \begin{macro}{\toc@pre@auto}
% \begin{macro}{\toc@post@auto}
% We patch \cmd\@starttoc\ to:
% 1) before TOC processing,
% initialize the max registers and
% set the needed dimensions from 
% the values stored in the auxiliary file, and
% 2) after TOC processing,
% store out those max register values into the auxiliary file.
%
% Note that the font is set here: all other TOC entries must
% override these font settings.
% 
% To activate this override of the standard \LaTeX\ processing,
% the substyle does: \cmd\let\cmd\toc@pre\cmd\toc@pre@auto\
% and \cmd\let\cmd\toc@post\cmd\toc@post@auto.
%    \begin{macrocode}
\def\toc@pre@auto{%
  \toc@@font
  \@tempdima\z@
  \toc@setindent\@tempdima{section}%
  \toc@setindent\@tempdima{subsection}%
  \toc@setindent\@tempdima{subsubsection}%
  \toc@setindent\@tempdima{paragraph}%
  \toc@letdimen{appendix}%
  \toc@letdimen{pagenum}%
}%
\def\toc@post@auto{%
  \if@filesw
   \begingroup
    \toc@writedimen{section}%
    \toc@writedimen{subsection}%
    \toc@writedimen{subsubsection}%
    \toc@writedimen{paragraph}%
    \toc@writedimen{appendix}%
    \toc@writedimen{pagenum}%
   \endgroup
  \fi
}%
%    \end{macrocode}
% \end{macro}
% \end{macro}
%
% \begin{macro}{\toc@setindent}
%    \begin{macrocode}
\def\toc@setindent#1#2{%
 \csname tocdim@#2\endcsname\tocdim@min\relax
 \@ifundefined{tocmax@#2}{\@namedef{tocmax@#2}{\z@}}{}%
 \advance#1\@nameuse{tocmax@#2}\relax
 \expandafter\edef\csname tocleft@#2\endcsname{\the#1}%
}%
%    \end{macrocode}
% \end{macro}
%
% \begin{macro}{\toc@letdimen}
%    \begin{macrocode}
\def\toc@letdimen#1{%
 \csname tocdim@#1\endcsname\tocdim@min\relax
 \@ifundefined{tocmax@#1}{\@namedef{tocmax@#1}{\z@}}{}%
 \expandafter\let\csname tocleft@#1\expandafter\endcsname\csname tocmax@#1\endcsname
}%
%    \end{macrocode}
% \end{macro}
%
% \begin{macro}{\toc@writedimen}
%    \begin{macrocode}
\def\toc@writedimen#1{%
 \immediate\write\@auxout{%
  \gdef\expandafter\string\csname tocmax@#1\endcsname{%
   \expandafter\the\csname tocdim@#1\endcsname
  }%
 }%
}%
%    \end{macrocode}
% \end{macro}
%
% \begin{macro}{\l@@sections}
% The procedure for formatting the indented TOC entries.
% We use control sequence names such as \cmd\tocmax@section\ and
% \cmd\tocleft@section, the former being written to the auxiliary file
% and the latter only defined for the duration of the TOC processing.
%
% Note that the assignment of \cmd\box\cmd\@tempboxa\ by \cmd\set@tocdim@pagenum\
% must endure over the invocation of |#3|: it contains the 
% page number which will be set just before the \cmd\par.
% 
% The arguments:\begin{enumerate}
% \item[\#1] superior section
% \item[\#2] this section
% \item[\#3] content, including possible \cmd\numberline
% \item[\#4] page number\end{enumerate}
%    \begin{macrocode}
\def\l@@sections#1#2#3#4{%
 \begingroup
  \everypar{}%
  \set@tocdim@pagenum\@tempboxa{#4}%
  \global\@tempdima\csname tocdim@#2\endcsname
  \leftskip\csname tocleft@#2\endcsname\relax
  \dimen@\csname tocleft@#1\endcsname\relax
  \parindent-\leftskip\advance\parindent\dimen@
  \rightskip\tocleft@pagenum plus 1fil\relax
  \skip@\parfillskip\parfillskip\z@
  \let\numberline\numberline@@sections
  \@nameuse{l@f@#2}%
  \ignorespaces#3\unskip\nobreak\hskip\skip@
  \hb@xt@\rightskip{\hfil\unhbox\@tempboxa}\hskip-\rightskip\hskip\z@skip
%    \end{macrocode}
% \changes{4.1n}{2009/12/13}{(AO, 574) protect against \classname{lineno.sty}, which forces a visit to the output routine, which appears to destroy the value of \cs{@tempdima}}%
%
% By side effect, set the value of, e.g., \cmd\tocdim@section. 
%
% Note that the \cmd\par\ must not be executed before the value of \cmd\@tempdima\ is expanded (outside the current group).
% Otherwise, the \classname{lineno.sty} package may interfere (it unfortunately does a global assignment of \cmd\@tempdima). 
%    \begin{macrocode}
  \expandafter\par
  \expandafter\aftergroup\csname tocdim@#2%
  \expandafter\endcsname
  \expandafter\endgroup
              \the\@tempdima\relax
}%
%    \end{macrocode}
% \changes{4.1a}{2008/01/19}{(AO, 479) Per: Dylan Thurston<dpt at math.harvard.edu>}%
% In the call to \cmd\set@tocdim@pagenum, I am now exposing the use of the particular box register. 
%    \begin{macrocode}
\def\set@tocdim@pagenum#1#2{%
 \setbox#1\hbox{\ignorespaces#2}%
 \@ifdim{\tocdim@pagenum<\wd#1}{\global\tocdim@pagenum\wd#1}{}%
}%
%    \end{macrocode}
% \end{macro}
%
% \begin{macro}{\numberline@@sections}
% \changes{4.1a}{2008/01/19}{(AO, 461) Change the csname from \cs{@dotsep} to \cs{ltxu@dotsep}. The former is understood in mu. (What we wanted was a dimension.)}%
% The utility procedure for all \cmd\numberline\ processing in indented TOC entries.
% The first argument is self.
%
% We use \cmd\@tempdima\ to pass a value around (via global assignment) because
% \cmd\numberline\ executes inside a group if the 
% \classname{hyperref} package is loaded.
% Would that it were not so!
%    \begin{macrocode}
\def\numberline@@sections#1{%
 \leavevmode\hb@xt@-\parindent{%
  \hfil
  \@if@empty{#1}{}{%
   \setbox\z@\hbox{#1.\kern\ltxu@dotsep}%
   \@ifdim{\@tempdima<\wd\z@}{\global\@tempdima\wd\z@}{}%
   \unhbox\z@
  }%
 }%
 \ignorespaces
}%
\def\tocdim@min{\z@}%
%    \end{macrocode}
% \end{macro}
%
%
% \subsection{Lists}
% \begin{macro}{\list}
% Using \cmd\parshape\ to implement lists was always suspect
% (can you get behind \cmd\parshape\cmd\@ne?) and we now see that
% it was a mistake all along. Why? Because \cmd\parshape, like
% \cmd\hangindent, achieves its effect via ``shifting'' the \cmd\hbox es
% in a paragraph
% instead of using \cmd\leftskip\ and \cmd\parindent, which is 
% robust during column balancing.
%
% We introduce the alternative method with a hook into
% the \LaTeX\ kernel procedure \cmd\list, which is
% the implementation of all lists. 
%
%    \begin{macrocode}
\def\list#1#2{%
  \ifnum \@listdepth >5\relax
    \@toodeep
  \else
    \global\advance\@listdepth\@ne
  \fi
  \rightmargin\z@
  \listparindent\z@
  \itemindent\z@
  \csname @list\romannumeral\the\@listdepth\endcsname
  \def\@itemlabel{#1}%
  \let\makelabel\@mklab
  \@nmbrlistfalse
  #2\relax
  \@trivlist
  \parskip\parsep
  \set@listindent
  \ignorespaces
}%
\def\set@listindent@parshape{%
 \parindent\listparindent
 \advance\@totalleftmargin\leftmargin
 \advance\linewidth-\rightmargin
 \advance\linewidth-\leftmargin
 \parshape\@ne\@totalleftmargin\linewidth
}%
\def\set@listindent@{%
 \parindent\listparindent
 \advance\@totalleftmargin\leftmargin
 \advance\rightskip\rightmargin
 \advance\leftskip\@totalleftmargin
}%
\let\set@listindent\set@listindent@parshape
%    \end{macrocode}
% \end{macro}
%
%
% \subsection{Hypertext capabilities}
%
% \begin{macro}{\href}
% \begin{macro}{\url}
% \begin{macro}{\URL@prefix}
% \begin{macro}{\doi}
% \begin{macro}{\doibase}
% \changes{4.1b}{2008/08/12}{(AO, 487) Support for video figures and the \cs{setfloatlink} command}%
% We provide support for the \cmd\href, \cmd\url, and \cmd\doi\ commands.
% Packages, like \classname{hyperref}, may override these definitions
% and provide better semantics.
% \changes{4.1g}{2009/10/06}{(AO, 532) Both arguments of \cs{href} get sanitized}%
% \changes{4.1j}{2009/10/24}{(AO, 545) Provide definition for \cs{doi} that does hypertext}%
%    \begin{macrocode}
\providecommand\href[0]{\begingroup\@sanitize@url\@href}%
\def\@href#1{\@@startlink{#1}\endgroup\@@href}%
\def\@@href#1{#1\@@endlink}%
\providecommand \url  [0]{\begingroup\@sanitize@url \@url }%
\def \@url #1{\endgroup\@href {#1}{\URL@prefix#1}}%
\providecommand \URL@prefix [0]{URL }%
\providecommand\doi[0]{\begingroup\@sanitize@url\@doi}%
\def\@doi#1{\endgroup\@@startlink{\doibase#1}doi:\discretionary {}{}{}#1\@@endlink }%
%changes{4.2a}{2017/11/21}{(MD) Use updated best practice to use https and doi.org}%
\providecommand \doibase [0]{https://doi.org/}%
\providecommand \@sanitize@url[0]{\chardef\cat@space\the\catcode`\ \@sanitize\catcode`\ \cat@space}%
%    \end{macrocode}
% \end{macro}
% \end{macro}
% \end{macro}
% \end{macro}
% \end{macro}
%
% \begin{macro}{\@@startlink}
% \begin{macro}{\@@endlink}
% \begin{macro}{\pdfstartlink@attr}
% \begin{macro}{\hypertext@enable@ltx}
% How we define \cmd\@@startlink\ and \cmd\@@endlink\ will depend on 
% whether we are running under \textsc{pdflatex}.
% If so, and if PDF output is requested, then we use its primitives
% to implement hypertext,
% breaking out the link attributes in \cmd\pdfstartlink@attr\ 
% and using the \classname{hyperref} defaults; 
% \cmd\pdfstartlink@attr\ can be redefined by a client package.
% Otherwise we fall back the Hyper\TeX\ standard and leave things to the DVI translator.
% \changes{4.1j}{2009/10/25}{(AO, 545) hypertext capabilities off by default; enable with \classoption{hypertext}}
%
% A class or package that wishes to employ hypertext capabilities should
% execute the \cmd\hypertext@enable@ltx\ procedure. 
%    \begin{macrocode}
\def\@@startlink#1{}%
\def\@@endlink{}%
\@ifxundefined \pdfoutput {\true@sw}{\@ifnum{\z@=\pdfoutput}{\true@sw}{\false@sw}}%
{%
 \def\@@startlink@hypertext#1{\leavevmode\special{html:<a href="#1">}}%
 \def\@@endlink@hypertext{\special{html:</a>}}%
}{%
 \def\@@startlink@hypertext#1{%
  \leavevmode
  \pdfstartlink\pdfstartlink@attr
   user{/Subtype/Link/A<</Type/Action/S/URI/URI(#1)>>}%
  \relax
 }%
 \def\@@endlink@hypertext{\pdfendlink}%
 \def\pdfstartlink@attr{attr{/Border[0 0 1 ]/H/I/C[0 1 1]}}%
}%
\def\hypertext@enable@ltx{%
 \let\@@startlink\@@startlink@hypertext
 \let\@@endlink\@@endlink@hypertext
}%
%    \end{macrocode}
% \end{macro}
% \end{macro}
% \end{macro}
% \end{macro}
%
% \begin{macro}{\href}
% \changes{4.1p}{2010/02/24}{(AO, 582) A patch of \classname{hyperref.sty} to provide backward compatibility to \TeX Live 2007's version 6.75r}%
% The \cmd\href\ command of \classname{hyperref} was extend somewhere 
% between versions 6.75r and 6.80e. We apply a repair to the earlier
% version (if present) so that it works like the later version.
%
% The issue is the presence of whitespace, either following the \cmd\href\ token
% or following the first argument's closing brace character. 
% 
%    \begin{macrocode}
\def\href@Hy{\hyper@normalise \href@ }%
\def\href@Hy@ltx{\@ifnextchar\bgroup\Hy@href{\hyper@normalise\href@}}%
\def\Hy@href#{\hyper@normalise\href@}%
\begingroup
  \endlinechar=-1 %
  \catcode`\^^A=14 %
  \catcode`\^^M\active
  \catcode`\%\active
  \catcode`\#\active
  \catcode`\_\active
  \catcode`\$\active
  \catcode`\&\active
  \gdef\hyper@normalise@ltx{^^A
    \begingroup
    \catcode`\^^M\active
    \def^^M{ }^^A
    \catcode`\%\active
    \let%\@percentchar
    \let\%\@percentchar
    \catcode`\#\active
    \def#{\hyper@hash}^^A
    \def\#{\hyper@hash}^^A
    \@makeother\&^^A
    \edef&{\string&}^^A
    \edef\&{\string&}^^A
    \edef\textunderscore{\string_}^^A
    \let\_\textunderscore
    \catcode`\_\active
    \let_\textunderscore
    \let~\hyper@tilde
    \let\~\hyper@tilde
    \let\textasciitilde\hyper@tilde
    \let\\\@backslashchar
    \edef${\string$}^^A
    \Hy@safe@activestrue
    \hyper@n@rmalise
  }^^A
  \catcode`\#=6 ^^A
  \gdef\Hy@ActiveCarriageReturn@ltx{^^M}^^A
  \gdef\hyper@n@rmalise@ltx#1#2{^^A
    \def\Hy@tempa{#2}^^A
    \ifx\Hy@tempa\Hy@ActiveCarriageReturn
      \Hy@ReturnAfterElseFi{^^A
        \hyper@@normalise{#1}^^A
      }^^A
    \else
      \Hy@ReturnAfterFi{^^A
        \hyper@@normalise{#1}{#2}^^A
      }^^A
    \fi
  }^^A
  \gdef\hyper@@normalise@ltx#1#2{^^A
    \edef\Hy@tempa{^^A
      \endgroup
      \noexpand#1{\Hy@RemovePercentCr#2%^^M\@nil}^^A
    }^^A
    \Hy@tempa
  }^^A
  \gdef\Hy@RemovePercentCr@ltx#1%^^M#2\@nil{^^A
    #1^^A
    \ifx\limits#2\limits
    \else
      \Hy@ReturnAfterFi{^^A
        \Hy@RemovePercentCr #2\@nil
      }^^A
    \fi
  }^^A
\endgroup
\def\switch@hyperref@href{%
 \expandafter\@ifx\expandafter{\csname href \endcsname\href@Hy}{
  \class@info{Repairing hyperref 6.75r \string\href}%
  \let\hyper@normalise\hyper@normalise@ltx
  \let\hyper@@normalise\hyper@@normalise@ltx
  \let\hyper@n@rmalise\hyper@n@rmalise@ltx
  \let\Hy@ActiveCarriageReturn\Hy@ActiveCarriageReturn@ltx
  \let\Hy@RemovePercentCr\Hy@RemovePercentCr@ltx
  \let\href\href@Hy@ltx
 }{}%
}%
\appdef\document@inithook{\switch@hyperref@href}%
%    \end{macrocode}
% \end{macro}
%
%
% \begin{macro}{\typeout}
% We make the \cmd\typeout\ procedure of \LaTeX\ be \cmd\long, 
% because sometimes we are talking about \cmd\par.
%    \begin{macrocode}
\def\typeout@org#1{%
 \begingroup
  \set@display@protect
  \immediate\write\@unused{#1}%
 \endgroup
}%
\long\def\typeout@ltx#1{%
 \begingroup
  \set@display@protect
  \immediate\write\@unused{#1}%
 \endgroup
}%
\@ifx{\typeout\typeout@org}{%
 \let\typeout\typeout@ltx
 \true@sw
}{%
 \rvtx@ifformat@geq{2020-10-01}%
   {\true@sw}{\false@sw}%
}%
 {\class@info{Making \string\typeout\space \string\long}}%
 {}%
%    \end{macrocode}
% \end{macro}
%
% \subsection{End of the \file{kernel} {\sc docstrip} module}
% Here ends the module.
%    \begin{macrocode}
%</kernel>
%    \end{macrocode}
%
% \Finale
% \iffalse Here ends the programmer's documentation.\fi
% \endinput
%
\endinput
%%EOF
